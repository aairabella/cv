%%%%%%%%%%%%%%%%%%%%%%%%%%%%%%%%%%%%%%%%%
% "ModernCV" CV and Cover Letter
% LaTeX Template
% Version 1.3 (29/10/16)
%
% This template has been downloaded from:
% http://www.LaTeXTemplates.com
%
% Original author:
% Xavier Danaux (xdanaux@gmail.com) with modifications by:
% Vel (vel@latextemplates.com)
%
% License:
% CC BY-NC-SA 3.0 (http://creativecommons.org/licenses/by-nc-sa/3.0/)
%
% Important note:
% This template requires the moderncv.cls and .sty files to be in the same 
% directory as this .tex file. These files provide the resume style and themes 
% used for structuring the document.
%
%%%%%%%%%%%%%%%%%%%%%%%%%%%%%%%%%%%%%%%%%

%----------------------------------------------------------------------------------------
%	PACKAGES AND OTHER DOCUMENT CONFIGURATIONS
%----------------------------------------------------------------------------------------

\documentclass[11pt,a4paper,sans]{moderncv} % Font sizes: 10, 11, or 12; paper sizes: a4paper, letterpaper, a5paper, legalpaper, executivepaper or landscape; font families: sans or roman

\moderncvstyle{casual} % CV theme - options include: 'casual' (default), 'classic', 'oldstyle' and 'banking'
\moderncvcolor{blue} % CV color - options include: 'blue' (default), 'orange', 'green', 'red', 'purple', 'grey' and 'black'

\usepackage{lipsum} % Used for inserting dummy 'Lorem ipsum' text into the template

\usepackage[scale=0.75]{geometry} % Reduce document margins

\hyphenation{miniMIPS}

%\setlength{\hintscolumnwidth}{3cm} % Uncomment to change the width of the dates column
\setlength{\makecvtitlenamewidth}{10cm} % For the 'classic' style, uncomment to adjust the width of the space allocated to your name

%\usepackage[spanish]{babel}
%\usepackage[utf8]{inputenc}

%----------------------------------------------------------------------------------------
%	NAME AND CONTACT INFORMATION SECTION
%----------------------------------------------------------------------------------------

\firstname{Andrés Miguel} % Your first name
\familyname{Airabella} % Your last name

% All information in this block is optional, comment out any lines you don't need
\title{Curriculum Vitae}
%\address{Barrancas Coloradas Neighborhood. M479 C9}{San Luis, Argentina}
\mobile{+54 9 266 4362456}
%\phone{(000) 111 1112}
%\fax{(000) 111 1113}
\email{a.airabella@gmail.com}
\homepage{http://about.me/andresmiguel}{about.me/andresmiguel} % The first argument is the url for the clickable link, the second argument is the url displayed in the template - this allows special characters to be displayed such as the tilde in this example
%\extrainfo{additional information}
%\photo[70pt][0.4pt]{pictures/picture} % The first bracket is the picture height, the second is the thickness of the frame around the picture (0pt for no frame)
%\quote{"Weniger, aber besser" - Dieter Rams}



%----------------------------------------------------------------------------------------

\begin{document}

%----------------------------------------------------------------------------------------
%	COVER LETTER
%----------------------------------------------------------------------------------------

% To remove the cover letter, comment out this entire block

%\clearpage

%\recipient{HR Department}{Corporation\\123 Pleasant Lane\\12345 City, State} % Letter recipient
%\date{\today} % Letter date
%\opening{Dear Sir or Madam,} % Opening greeting
%\closing{Sincerely yours,} % Closing phrase
%\enclosure[Attached]{curriculum vit\ae{}} % List of enclosed documents
%
%\makelettertitle % Print letter title
%
%\lipsum[1-2] % Dummy text
%\lipsum[4] % Dummy text
%
%\makeletterclosing % Print letter signature
%
%\newpage

%----------------------------------------------------------------------------------------
%	CURRICULUM VITAE
%----------------------------------------------------------------------------------------

\makecvtitle % Print the CV title

%----------------------------------------------------------------------------------------
%	EDUCATION SECTION
%----------------------------------------------------------------------------------------

\section{Contact Information}

\cvitem{Full Name}{Andrés Miguel Airabella}
%\cvitem{Date of Birth}{June 4th, 1984}
%\cvitem{Place of Birth}{San Rafael, Mendoza, Argentina}
%\cvitem{Nationality}{Argentinian}
%\cvitem{Civil State}{Married. Three kids.}
%\cvitem{Postal Address}{Barrancas Coloradas Neighborhood, Manzana 479, Casa 9. 5700 - San Luis. San Luis - Argentina}

\cvitem{Phone}{+54-9-266-4362456}
\cvitem{Email}{a.airabella@gmail.com}


\section{Professional Summary}

\cvitem{}{As an ambitious yet humble professional, I have a great passion for continuous learning and personal development. I have experience in digital electronics, signal acquisition and processing and power electronics. Nowadays, I am particularly interested in project management and project execution, as well as product development and customer support. My goal is to acquire strong skills in these areas and effectively apply them to achieve exceptional results. Since the beginning of my career, I have worked in multidisciplinary teams distributed across the globe in different time zones with a rich cultural diversity. The challenge has been immense but very enriching. I am committed to professional growth and am willing to take on challenges to expand my knowledge and abilities in new fields.}


\section{Work Experience in the Private Sector}

\cvitem{2021-present}{January 2021 to present: \textbf{Satellogic S.A. Payload System Flight Engineer.} In this position, I am responsible for the commissioning, calibration, and maintenance of image sensors on a fleet of over 40 satellites in orbit. My main achivements in this role include: I managed to cut in half, from 80 to 40 days, the time required to deploy payloads in orbit, increasing the availability of our satellites and extending their orbital usage time. I trained operators and provided them with procedures for payload initialization. I helped transform all our procedures into maintenance tools, which accelerated the deployment processes and significantly reduced satellite downtime for maintenance. Additionally, I implemented systems to assess the quality of images generated by our payloads, helping to decrease the calibration time from 4 weeks to 1 week.
}
\cvitem{2017--2020}{May 2017 to December 2020: \textbf{Satellogic S.A. HDL (Hardware Description Language) \& Electronics Development Senior Engineer.} In this role, I participated in the design of payloads for Earth observation satellites. My main achivements in this role include: Working on the design of boards for image sensors, their interfaces, and power supplies. I designed IP Cores in HDL (Verilog and VHDL). I worked on the validation cycle and commissioning of the boards prior to flight. I designed procedures to expedite the deployment of payloads and their integration in the factory.}
\cvitem{2016--2017}{August 2016 to April 2017: \textbf{Advisor to the Government of the Province of San Luis} on Renewable Energy and Energy Efficiency. Renewable Energy and Energy Efficiency Program, Ministry of Environment, Field, and Production. San Luis, Argentina.}
\cvitem{2014--2017}{Dec 2014 to May 2017: \textbf{Enertronics AR. General Manager.} In this role, I founded and managed the company Enertronics AR. The product portfolio included Drivers (Activation Circuits) for Hi Power Transistors like MOSFETs and IGBT used in applications such as renewable energies and electric vehicles. The company also developed several power converters, like boost converters for solar array managements, H-bridges for AC Inverters, and three-phase inverters for electrical machines drives.}
\cvitem{2009--2010}{June 2009 to April 2010: \textbf{Scholarship holder at the Environmental Studies Group}, providing support in the development of electronic equipment for environmental variable measurements.}
\cvitem{2008--2009}{October 2008 to March 2009: \textbf{Junior Electronics Engineer and HDL Designer.} Design of low-power digital modules/macros (IP Cores) using hardware description languages for MGB Design SRL, subcontracted to Actel Corp (now Microchip).}


\section{Professional Training}

\subsection{Formal Studies}
\cventry{2011--(Incomplete)}{PhD in Engineering Sciences}{National University of Río Cuarto}{Faculty of Engineering}{Thesis Topic: Fault-Tolerant DC-DC Converters for Hybrid Electric Systems Applications}{My studies work included the evaluation of fault tolerant topologies for DC-DC converters, the design of power electronics circuit drivers and controllers using DSP.}

\cventry{2003--2009}{Electronics Engineering with a focus on Digital Systems (OSD)}{National University of San Luis}{Faculty of Physics Maths and Natural Sciences}{Average: 8.83 (eight point eight three), with and without failed courses}{My thesis was about an Image Processing platform using FPGA.}

\cventry{1997--2001}{Accounting and Administrative Technicature with a focus on computer science}{San Rafael School, 28 P.S. Marist Brothers}{San Rafael, Mendoza}{}{}

\subsection{Languages}

\cvitem{English}{Level 4 completed and approved at I.A.I. Language School, San Rafael, Mendoza. English I and II, corresponding to the Electronics Engineering with OSD program. Excellent speaking, reading, text interpretation, and writing skills, equivalent to C1 Level.}
\cvitem{Italian}{Level A2 completed at Danthe Alighieri School, Villa Mercedes, San Luis.}
\cvitem{Spanish}{Native language.}
\subsection{Postgraduate Courses}

\cvitem{2012}{MODELING AND CONTROL OF ELECTRIC MACHINES (Ding-14). Organized by the Faculty of Engineering, National University of Río Cuarto. Instructors: Dr. Eng. Guillermo O. García and Dr. Eng. Pablo M. De la Barrera. Duration: 60 hours. Dates: April 1, 2012, to July 10, 2012. Certificate: May 27, 2013.}
\cvitem{2012}{SWITCHED CONVERTERS (Code 2467). Organized by the Institute of Electrical Engineering Research, National University of Sur. Responsible Instructors: Drs. A. Oliva - M. D'Amico. Duration: 80 hours. Approved on June 28, 2012. Certificate: October 30, 2012.}
\cvitem{2010}{POWER ELECTRONICS (Dinge-06). Organized by the Faculty of Engineering, National University of Río Cuarto. Instructors: Dr. Eng. Guillermo O. García and Dr. Eng. Germán Oggier. Duration: 60 hours. Dates: September 1, 2010, to December 17, 2010. Certificate: March 20, 2012.}
\cvitem{2012}{ADVANCED LINEAR CONTROL. Organized by the Faculty of Engineering, National University of Río Cuarto. Instructors: Dr. Eng. Cristian de Ángelo, assistant: Ms. Lic. Laura Pérez. Duration: 60 hours. Dates: September 11, 2012, to December 11, 2012. Certificate: May 27, 2013.}
\cvitem{2012}{INSTANTANEOUS POWER THEORY AND APPLICATIONS. National University of Río Cuarto. Second semester 2012. Responsible: Dr. Eng. G. O. García and Dr. Eng. G.G. Oggier. Duration: 60 hours. Approved: November 3, 2014.}
\cvitem{2013}{POWER ELECTRONICS IN MICROGRIDS WITH HIGH PENETRATION OF RENEWABLE ENERGY. (Ding-26) National University of Río Cuarto. September 3 to December 30, 2013. Responsible: Dr. Eng. Denizar C. Martins and Dr. Eng. Guillermo. O. García. Duration: 60 hours. Approved. Certificate: December 29, 2014.}
\cvitem{2015}{LABORATORY OF IMPLEMENTATION OF EXPERIMENTAL PROTOTYPES. National University of Río Cuarto. Responsible: Dr. Eng. Germán G. Oggier and Dr. Eng. Guillermo. O. García. Duration: 80 hours. Approved: 2015.}

\subsection{Postgraduate Seminars}

\cvitem{2012}{RESEARCH SEMINAR IN CONTROL AND ENERGY CONVERSION. Organized by the Faculty of Engineering, National University of Río Cuarto. Duration: 40 hours. Dates: April 25, 2012, to December 10, 2012. Certificate: May 23, 2013.}
\cvitem{2011}{RESEARCH SEMINAR IN ENGINEERING SCIENCES. Organized by the Faculty of Engineering, National University of Río Cuarto. Duration: 40 hours. Dates: April 6, 2011, to September 6, 2011. Certificate: November 20, 2012.}
\cvitem{2011}{RESEARCH SEMINAR IN CONTROL AND ENERGY CONVERSION. Organized by the Faculty of Engineering, National University of Río Cuarto. Dates: March 23, 2011, to December 18, 2011. Duration: 40 hours. Certificate: May 11, 2012.}
\subsection{Training and Development Courses}

\cvitem{2018}{Course: "Advanced Design of Embedded Systems on Programmable Logic: Zynq APSoC, Vivado HLS, and SDSoC". Credit Hours: 60 hours. LEIS. Electra Training. March 19 to 23, 2018. National University of San Luis. San Luis, Argentina.}
\cvitem{2017}{Course: "High-Level Synthesis for FPGAs with Vivado-HLS". LEIS. Electra Training. April 11 and 12, 2017. LEIS. National University of San Luis. San Luis, Argentina.}
\cvitem{2017}{Course: Accelerating FPGA Design (FPGA Competence). Bitbis AS (Norway). May 8 and 9, 2017. Satellogic S.A. Buenos Aires, Argentina.}
\cvitem{2017}{Course: Advanced VHDL Verification - Made Simple (FPGA Competence). Bitbis AS (Norway). May 9 to 11, 2017. Satellogic S.A. Buenos Aires, Argentina.}
\cvitem{2009}{Course: "Advanced Training Course on FPGA Design and VHDL for Hardware Simulation and Synthesis" at the International Centre for Theoretical Physics (ICTP) in Trieste, Italy. Duration: 120 clock hours. October 2009.}

\subsection{Undergraduate Teaching}

\cvitem{2019--present}{Assistant Professor, Part-time. Faculty of Physics, Maths and Natural Sciences. National University of San Luis. Date: February 5, 2019, to present. Responsible for courses: Computer Architecture (Electronics Engineering with a focus on Digital Systems), Audio and Video (Telecommunications Technicature), and Multimedia Production (Telecommunications Technicature).}
\cvitem{2017--2019}{Assistant Professor, Full-time. Faculty of Physics, Maths and Natural Sciences. National University of San Luis. Date: September 1, 2017, to February 4, 2019. Responsible for courses: Computer Architecture (Electronics Engineering with a focus on Digital Systems), Audio and Video (Telecommunications Technicature), and Multimedia Production (Telecommunications Technicature).}
\cvitem{2016--2017}{Assistant Professor. Full Time.  Faculty of Physics, Maths and Natural Sciences. National University of San Luis. Date: April 14, 2016, to August 30, 2017. Responsible for courses: Computer Architecture (Electronics Engineering with a focus on Digital Systems), Audio and Video (Telecommunications Technicature), and Multimedia Production (Telecommunications Technicature).}
\cvitem{2013--2016}{Assistant Professor, Part-time, Temporary Character. Faculty of Physics, Maths and Natural Sciences. National University of San Luis. Date: July 22, 2013, to April 13, 2016. Responsible for courses: Computer Architecture (Electronics Engineering with a focus on Digital Systems), Audio and Video (Telecommunications Technicature), and Multimedia Production (Telecommunications Technicature).}
\cvitem{2012--present}{First-level Teaching Assistant, Part-time dedication. Faculty of Physics, Maths and Natural Sciences. National University of San Luis. Date: November 1, 2012, to present. On leave without pay due to incompatibility. Responsible for courses: Computer Architecture (Electronics Engineering with a focus on Digital Systems), Audio and Video (Telecommunications Technicature), and Multimedia Production (Telecommunications Technicature).}
\cvitem{2011--2012}{First-level Teaching Assistant, Part-time. Faculty of Physics, Maths and Natural Sciences. National University of San Luis. Date: April 1, 2011, to October 30, 2012. Responsible for Practical Works: Computer Architecture (Electronics Engineering with a focus on Digital Systems) and Digital Systems Design (Electronics Engineering with a focus on Digital Systems).}
\cvitem{2010--2011}{First-level Teaching Assistant. Full-Time. Faculty of Physics, Maths and Natural Sciences. National University of San Luis. Date: March 12, 2010, to March 31, 2011. Responsible for Practical Works: Computer Architecture (Electronics Engineering with a focus on Digital Systems) and Digital Systems Design (Electronics Engineering with a focus on Digital Systems).}
\cvitem{2009-2010}{Second-level Teaching Assistant, Part-time. Faculty of Physics, Maths and Natural Sciences. National University of San Luis. Date: June 9, 2009, to March 11, 2010. Teaching Assistant: Computer Architecture (Electronics Engineering with a focus on Digital Systems) and Digital Systems Design (Electronics Engineering with a focus on Digital Systems).}

\subsection{Postgraduate Teaching}

\cvitem{2013}{POWER ELECTRONICS. Taught by the Faculty of Mathematical and Natural Sciences, National University of San Luis. Instructors: Dr. Ing. Germán Oggier. Teaching Assistant: Ing. Andrés Miguel Airabella. Coordinator: Ing. Cristian Ariel Falco. Duration: 60 hours. Dates: November 1, 2013, to November 30, 2013.}

\subsection{Bachelor's Thesis Evaluation}

\cvitem{2023}{"Crono TDC: Design and implementation of a Time to Digital Converter on FPGA." Author: Julián Rodriguez. Supervisor: Eng. Nicolás Alvarez. Co-Supervisor: Dr. Federico Izraelevitch. Degree: Electronics Engineering. School of Science and Technology. UNSAM. May 31, 2023.} 
\cvitem{2022}{Obstacle detection on sidewalks for the assistance of visually impaired individuals." Author: Lucas Carranza. Supervisor: Dr. Emanuel Trabes. Degree: Electronics Engineering with a focus on Digital Systems. UNSL. November 4, 2022.}
\cvitem{2022}{"Detection and classification of cracks in asphalt pavements using Computer Vision." Author: Emanuel Alfredo Cortez Médici. Supervisor: Eng. Ricardo Petrino. Degree: Electronics Engineering with a focus on Digital Systems. UNSL. December 1, 2022.} 
\cvitem{2016}{Wireless networks for the development of digital rural communities in the Yauya district." Author: Ada Luz Caballero Sifuentes. Supervisor: Eng. Alfredo Debattista. Degree: Electronics Engineering with a focus on Digital Systems. UNSL. December 2016.}
\cvitem{2016}{"Load cell measurement system with AC excitation based on a lock-in amplifier." Author: Laura Beatriz Adaro. Supervisor: Carlos Federico Sosa Páez. Degree: Electronics Engineering with a focus on Digital Systems. UNSL. December 2016.} 
\cvitem{2011}{Acoustic signal processing using programmable logic." Author: Felix Leonardo Garro Martínez. Supervisor: Diego Esteban Costa. Degree: Electronics Engineering with a focus on Digital Systems. UNSL. September 16, 2011.}
\cvitem{2011}{"Implementation of basic functions of the lock-in amplifier on FPGA." Author: Layla María Martínez Guevara. Supervisor: Eng. Esteban Maximiliano Peláez. Degree: Electronics Engineering with a focus on Digital Systems. UNSL. December 19, 2011.}

\subsection{Supervision of Scholarships and Final Projects}

\cvitem{2017-2018}{Final Career Project: Student: Emiliano Álvarez. Title of the Work Plan: Design of a Test Bench for Power Transistor Activation Circuits (Drivers). Academic Year 2016-2017. Completed. Role: Supervisor.}
\cvitem{2016--2017}{Final Career Project: Student: Rodrigo Agustn Perna. Title of the Work Plan: Supervision and Control System on FPGA for a Solar Battery Charger. Academic Year 2016-2017. Completed. Role: Supervisor.}
\cvitem{2016-2017}{CIN (National Interuniversity Council) Research Scholarship: Student: Rodrigo Agustín Perna. Title of the Work Plan: Supervision and Control System on FPGA for a Solar Battery Charger. Academic Year 2016-2017. Completed. Role: Supervisor.}
\cvitem{2012--2014}{TIC 2012 Scholarship and Final Career Project: Student: María Julia Xacur. Title of the Work Plan: Development of Specific Instrumentation for Measuring Breaking Force of Solids. Academic Year 2012-2014. Completed.}
\cvitem{2012--2013}{TIC 2012 Scholarship: Student: Gerardo Galo. Title of the Work Plan: Design of a Low-Power LED Lighting Source. Academic Year 2012-2013. Completed.}
\cvitem{2012--2013}{Co-Supervision of Final Career Project: Student: Enzo Belpoliti. Title of the Work Plan: Implementation of the SVPWM Modulation Technique for Control of Three-Phase Inverters. Academic Year 2012-2013. Completed.}

\subsection{Supervision of Research Stays}

\cvitem{2012}{Supervision of the "Summer of Science, Universidad Autónoma San Luis Potosí 2012" Scholarship. Student: Oscar Zamarripa. Title of the Work: "Operation of a DC-DC Converter with Active Dual Bridges in Forward Mode for the Study of Semiconductor Faults." Location: "Applied Electronics Group - Faculty of Engineering - National University of Río Cuarto." Mexico-Argentina Exchange Scholar. May to June 2012.}
\cvitem{2013--2014}{Supervision of the "UNSL-Hochschule Bonn Rhein Sieg" Exchange Scholarship. Student: Sven Stockhausen. Title of the Work: "Design of an Instrument for Measuring Breaking Force of Solids." Location: Laboratory of Electronics, Research and Services - National University of San Luis. Germany-Argentina Exchange Scholar. October 2013 to February 2014.}


\section{Background in Scientific Research and Technological Development}

\subsection{Participation in Research and Development Programs and Projects}
\cvitem{2016}{University extension project: "Worker-Recuperated Factories." Accepted through CS 279-16 resolution, National University of San Luis. Role: Member.}
\cvitem{2015--2016}{"Hardness Measurement Equipment." Project of the University, Design, and Productive Development Program 2015. Role: Project Director.}
\cvitem{2015--2016}{"Intelligent Energy Meter." Project of the University, Design, and Productive Development Program 2015. Role: Project Director.}
\cvitem{2015--2017}{(Biennial project) PDTS-CIN-CONICET PDTS209. "Modular Traction System for Electric Vehicles." Principal Investigator: Cristian De Angelo, Area: Engineering and Technology. National University of Río Cuarto. CIN-CONICET, Resolution CE Nº 1055-15.}
\cvitem{2015--2018}{(Three-year project) PICT-2014-2760. "Electric Traction Urban Vehicles: Propulsion System and Energy Management." Principal Investigator: Cristian De Angelo, Area: Mining Mechanical and Materials Energy Technology. National University of Río Cuarto. ANPCyT, Resolution Nº 270-15.}
\cvitem{2015--2018}{(Three-year project) CONICET PIP 2014-2016 GI "Electric Traction Urban Vehicles: control, supervision, energy management, and integration into the power grid." Director: Guillermo O. García, Co-director: Cristian H. De Angelo. Approved. Res. 5013/14.}
\cvitem{2015--2018}{(Three-year project) PICT-Start-Up 2014-3647 "R\&D of Technologies for High-Power UPS." Director: Guillermo O. García. Joint project between CREXEL S.A. and GEA-UNRC.}
\cvitem{2014--2015}{(Biennial project) PROIPRO 142514. "Power Converters Control for Renewable Energy Systems." Financed by the National University of San Luis. Work line: "Programmable Logic Applied to Power Electronics." Project Director: Cristian De Angelo, Line Director: Cristian Ariel Falco.}
\cvitem{2014--2017}{(Three-year project) PICT-2013-1194 "Parallelism of Three-Phase Inverters for Renewable Energy Integration in Microgrids" Open Topics. Guillermo Oscar García, Energy, Mining, Mechanical, and Materials Technology, National University of Río Cuarto.}
\cvitem{2014--2015}{"Driver Testing Bench." Project of the University, Design, and Productive Development Program 2014. Role: Project Director.}
\cvitem{2014--2015}{"PLD in Power Electronics." Project of the University, Design, and Productive Development Program 2014. Role: Collaborating Professor.}
\cvitem{2012--2013}{"Alternative Energies for Electric Power Supply in the Central West Region." ANPCyT FONCyT CIN II PICTO-2010-0154 project.}
\cvitem{2011--2013}{"Electro-Eolic Generation Systems - 5 and 30 kW Scales." Joint project between FONCyT, ANPCyT, PICT-2010.}
\cvitem{2011--2013}{"Detection and Diagnosis of Incipient Faults in Electric Drives." PIP CONICET 2010.}
\cvitem{2013--2014}{"Fault-Tolerant DC Power Flow Control." Project subsidized by the Ministry of Science and Technology of the Province of Córdoba. Res. Nº 113/2011. Director: Dr. Eng. Germán Gustavo Oggier.}
\cvitem{2012--2015}{"Control and Energy Conversion Program." Research Program subsidized by the Secretariat of Science and Technology of UNRC. Res. Rec. 328/12. The Program includes the following Projects: Electro-Eolic Generation Systems, Electric and Hybrid Vehicles, Diagnosis of Incipient Faults in Electric Drives.}
\cvitem{2012}{International Cooperation Project "Power Electronics Topologies for Efficient Integration of Renewable Energy Sources into Microgrids and Electrical Systems," subsidized by the Ministry of Science and Technology of the Province of Córdoba. Joint project between GEA-UNRC, Argentina, and INEP-UFSC, Brazil. Project CCII -11347549-A-2011. Res. 174/11.}
\cvitem{2009--2011}{"Control and Energy Conversion Program." Research Program subsidized by the Secretariat of Science and Technology of UNRC. Res. Rec. 442/09. The Program includes the following Projects: Electro-Eolic Generation Systems, Electric and Hybrid Vehicles, Diagnosis of Incipient Faults in Electric Drives.}

\subsection{Received Scholarships}
\cvitem{2014--2015}{Internal Postgraduate Training Scholarship Type II. Project: "Fault-Tolerant DC-DC Converters for Hybrid Electric Systems." Awarded by the National Council for Scientific and Technical Research (CONICET). Duration: April 2014 to March 2016.}
\cvitem{2011--2013}{Internal Postgraduate Training Scholarship Type I. Project: "Fault-Tolerant DC-DC Converters for Hybrid Electric Systems." Awarded by the National Council for Scientific and Technical Research (CONICET). Duration: April 2011 to March 2014.}
\cvitem{2009}{International Center for Theoretical Physics (ICTP). Full Scholarship for the course "Advanced Training Course on FPGA Design and VHDL for Hardware Simulation and Synthesis." Trieste, Italy. October and November 2009.}
\cvitem{2007--2008}{Stimulus Scholarship from the Faculty of Physics, Maths, and Natural Sciences. Research Work (Final Degree Project): "Image Processing on FPGA."}

\section{Scientific Research Production - Publications}

\subsection{Published Papers in Indexed International Journals}

\cvitem{2015}{\textbf{A.M. Airabella}, G.G. Oggier, L.E. Piris-Botalla, C. A. Falco, and G.O. García, "Semiconductors Faults Analysis in Dual Active Bridge DC-DC Converter." IET Power Electronics. October 2015. DOI: 10.1049/iet-pel.2015.0299, Online ISSN 1755-4543}
\cvitem{2014}{L. E. Piris Botalla, G. G. Oggier, \textbf{A. M. Airabella}, and G. O. García, "Power Losses Evaluation of a Bidirectional Three-Port DC-DC Converter for Hybrid Electric System." Elsevier International Journal of Electrical Power \& Energy Systems (ISSN: 0142-0615). Volume 58, June 2014, Pages 1-8}
\cvitem{2016}{L. E. Piris Botalla, G. G. Oggier, \textbf{A. M. Airabella}, and G. O. García, "Extension of the Operating Range with Soft Switching of a Bidirectional Three-Port DC-DC Converter." Revista Iberoamericana de Automática e Informática Industrial (RIAI) (ISSN: 1697-7912) - Accepted March 2015.}

%\subsection{Papers in Press in Indexed International Journals}

%\subsection{Papers Under Evaluation in Indexed International Journals}

\subsection{Published Papers in Non-Indexed National Journals}

\cvitem{2015}{\textbf{A. M. Airabella}, G. G. Oggier, L. E. Piris-Botalla, Cristian A. Falco, and G. O. García, "Fault Detection Strategy for Open Circuit Faults in Semiconductors of Isolated DC-DC Converters." Nota T'ecnica, Revista Ingeniería El'ectrica. Editores On-Line S.R.L. March 2015.}

%\subsection{Papers Under Evaluation in National Refereed Conferences}

\subsection{Full Papers in International Refereed Conferences}

\cvitem{2023}{\textbf{A. M. Airabella.} "Ambitious: Edge AI Platform." XII Argentine Congress of Space Technology. April 12-14, 2023. Mendoza, Argentina.}
\cvitem{2019}{\textbf{A. M. Airabella}, D. Caruso, A. J. Demski. "BitSync: A novel Data to Clock Phase Alignment for Microsemi FPGAs." SPL Conf. April 10th to 12th, 2019. Buenos Aires, Argentina.}
\cvitem{2019}{A. J. Demski, D. Caruso, \textbf{A. M. Airabella}. "Clock Domain Crossing Problem Solution Proposal in an AXI to AHB IP Core." SPL Conf. April 10th to 12th, 2019. Buenos Aires, Argentina.}
\cvitem{2019}{D. Caruso, \textbf{A. M. Airabella}, R. A. Melo. "High-Speed serial protocol multi-link and multi-stage for FPGAs." SPL Conf. April 10th to 12th, 2019. Buenos Aires, Argentina.}
\cvitem{2018}{\textbf{A. M. Airabella}, C. Falco, G. Oggier, and G. O. Garcia. "Fault Detection, Diagnosis, and Tolerance for Transistors in DC-DC Converters with Dual Active Bridges in Buck Mode." IEEE ARGENCON 2018. June 6-8, 2018. San Miguel de Tucumán.}
\cvitem{2012}{\textbf{A. M. Airabella}, G. G. Oggier, L. E. Piris-Botalla, Cristian A. Falco, and G. O. García, "Open Transistors and Diodes Fault Diagnosis Strategy for Dual Active Bridge DC-DC Converter." 10th IEEE/IAS International Conference on Industry Applications (INDUSCON 2012). November 5-7, 2012, in Fortaleza.}
\cvitem{2012}{L.E. Piris-Botalla, G. G. Oggier, \textbf{A. M. Airabella}, and G. O. García. "Analysis and Evaluation of Power Switch Losses for Three-Port Bidirectional DC-DC Converter." 2012 IEEE International Conference on Industrial Technology (ICIT), Athens, Greece, March 19-21, 2012.}

\subsection{Full Papers in National Refereed Congresses}

\cvitem{2017}{M. E. Postemsky, S. F. Hernández Velázquez, R. M. Murdocca, \textbf{A. M. Airabella}. "Intelligent Energy Meter". VIII Congress of Applied Microelectronics. Córdoba, October 11-13, 2017.}
\cvitem{2016}{R. A. Perna, O. E. Álvarez, J. R. Sánchez, and \textbf{A. M. Airabella}. "Control Logic of a Test Bench for Power Transistor Activation Circuits". VII Congress of Applied Microelectronics 2016. Universidad Nacional de San Luis, San Luis, Argentina.}
\cvitem{2016}{O. E. Álvarez, R. A. Perna, J. R. Sánchez, and \textbf{A. M. Airabella}. "Implementation of a Test Bench for Power Transistor Activation Circuits". VII Congress of Applied Microelectronics 2016. Universidad Nacional de San Luis, San Luis, Argentina.}
\cvitem{2016}{\textbf{A. M. Airabella}, G. G. Oggier, L. E. Piris-Botalla, C. A. Falco, G. O. García. "Power Transfer Limits of a Fault-Tolerant Step-Down Dual Active Bridge DC-DC Converter". XVI Meeting on Information Processing and Control (XVI RPIC) Ciudad de Córdoba, October 6-8, 2015.}
\cvitem{2015}{L. E. Piris-Botalla, G. G. Oggier, \textbf{A. M. Airabella}, and G. O. García. "Design of Auxiliary Inductances for a Bidirectional Three-Port DC-DC Converter". XVI Meeting on Information Processing and Control (XVI RPIC) Ciudad de Córdoba, October 6-8, 2015.}
\cvitem{2015}{L. E. Piris-Botalla, G. G. Oggier, \textbf{A. M. Airabella}, and G. O. García. "Comparison of Losses in a Bidirectional Three-Port DC-DC Converter for Different Values of Auxiliary Inductances". XVI Meeting on Information Processing and Control (XVI RPIC) Ciudad de Córdoba, October 6-8, 2015.}
\cvitem{2015}{L. E. Piris-Botalla, G. G. Oggier, \textbf{A. M. Airabella}, and G. O. García. "Determination and Extension of the Operating Limits of a Bidirectional Three-Port DC-DC Converter". XVI Meeting on Information Processing and Control (XVI RPIC) Ciudad de Córdoba, October 6-8, 2015.}
\cvitem{2014}{E. Belpoliti, \textbf{A. M. Airabella}, Cristian Ariel Falco. "SVPWM Technique in FPGA for Three-Phase Inverter Control". 24th Argentine Congress on Automatic Control. October 27-29, 2014 - Buenos Aires, Argentina.}
\cvitem{2013}{\textbf{A. M. Airabella}, G. G. Oggier, L. E. Piris-Botalla, C. A. Falco, G. O. García. "Fault-Tolerant Power Transistor Schemes for Dual Active Bridge Converters". XV Meeting on Information Processing and Control (XV RPIC) S.C. de Bariloche, September 16-20, 2013.}
\cvitem{2011}{\textbf{A. M. Airabella}, G. G. Oggier, L. E. Piris-Botalla, Cristian A. Falco, and G. O. García. "Fault Detection Strategy for Open Circuit Faults in Isolated DC-DC Converters". AADECA 2012, October 3-5, 2012, Buenos Aires, Argentina.}
\cvitem{2011}{A. I. Testa, M. R. Palavecino Nicotra, \textbf{A. M. Airabella}, F. Aguilera. "Implementation of an Embedded System on FPGA Based on the miniMIPS Microprocessor". III Congress of Applied Microelectronics 2012. Student Section. 2012. ISBN 978-987-702-004-5.}
\cvitem{2011}{\textbf{A. M. Airabella}, G. G. Oggier, L. E. Piris-Botalla, and G. O. García. "Fault Diagnosis in Power Semiconductor Devices of Dual Active Bridge DC-DC Converters". XIV Meeting on Information Processing and Control (RPIC 2011), Universidad Nacional de Entre Ríos, Oro Verde, Entre Ríos, Argentina, November 16-18, 2011.}
\cvitem{2011}{L. E. Piris-Botalla, G. G. Oggier, \textbf{A. M. Airabella}, and G. O. García. "Bidirectional Three-Port DC-DC Converter: Evaluation of Power Switches Losses". XIV Meeting on Information Processing and Control (RPIC 2011), Universidad Nacional de Entre Ríos, Oro Verde, Entre Ríos, Argentina, November 16-18, 2011.}
\cvitem{2010}{\textbf{A. M. Airabella}, C. Sosa Páez, R. Petrino. "Platform for Image Processing on FPGA". I Congress of Applied Microelectronics, UnLAM 2010.}

\section{Participation in Scientific and Educational Meetings}

\subsection{As an Author of Papers}

\cvitem{2013}{XVI Meeting on Information Processing and Control (XVI RPIC). Title of the paper: "Power Transfer Limits of a Fault-Tolerant Step-Down Dual Active Bridge DC-DC Converter". City of Córdoba, October 6-8, 2015.}
\cvitem{2012}{AADECA'12 - Week of Automatic Control. 23rd Argentine Congress of Automatic Control. October 3-5, 2012. Buenos Aires, Argentina. Title of the paper: "Fault Detection Strategy for Open Circuit Faults in Isolated DC-DC Converters".}
\cvitem{2011}{XIV Meeting on Information Processing and Control (RPIC 2011), National University of Entre Ríos, Oro Verde, Entre Ríos, Argentina. November 16-18, 2011. Title of the paper: "Fault Diagnosis in Power Semiconductor Devices of Dual Active Bridge DC-DC Converters".}
\cvitem{2010}{I Congress of Applied Microelectronics. 2010. La Matanza, Buenos Aires, Argentina. Title of the paper: "Platform for Image Processing on FPGA".}

\subsection{As a Speaker, Teacher, or Organizer}

\cvitem{2018}{"Satellogic: A talk about satellites". March 21, 2018, National University of San Luis. Organized by IEEE San Luis Student Branch.}
\cvitem{2017}{SASE 2017 - Speaker at the Workshop: "Starting Point for FPGA-VHDL Designs" in the Argentine Symposium on Embedded Systems (SASE 2017), August 9-11, 2017.}
\cvitem{2016}{uEA 2016 - VII Applied Microelectronics Congress 2016. National University of San Luis. October 26-28, 2016, San Luis, Argentina.}
\cvitem{2015}{III JOREIC - 3rd Regional Conference of Civil Engineering Students 2015. "Do you have two minutes? How to share ideas in a limited time". National Technological University. Regional Faculty San Rafael. May 22, 2015.}
\cvitem{2014}{XX JOSEII. Title of the conference: "Do you have two minutes? How to share ideas in a limited time". National Technological University. Regional Faculty San Rafael. November 7, 2014.}
\cvitem{2014}{Title of the conference: "Prohibited to Look for a Job". National University of Cuyo, Faculty of Applied Sciences to Industry. Organized by AECA. San Rafael. November 4, 2014.}
\cvitem{2014}{II National Week of Technological Entrepreneur. Title of the conference: "Prohibited to Look for a Job". National Technological University. Regional Faculty San Rafael. September 16, 2014.}
\cvitem{2012}{Seventh National Meeting of Engineering Students (ENEI 2012). Title of the conference: "Engineer? Prohibited to Look for a Job". August 18-19, 2012. San Luis, Argentina.}
\cvitem{2011}{Argentine Symposium on Embedded Systems 2011. Buenos Aires, Argentina. Title of the conference: "Advanced Simulation with TestBench in HDL".}
\cvitem{2010}{Fifth National Meeting of Engineering Students (ENEI 2010). Title of the conference: "Engineer? Prohibited to Look for a Job". November 25-27, 2010, Río Cuarto, Córdoba, Argentina.}
\cvitem{2010}{I Congress of Applied Microelectronics. 2010. La Matanza, Buenos Aires, Argentina. Title of the conference: "Advanced Simulation with TestBench in HDL".}
\cvitem{2010}{Seminar "Introduction to Programmable Logic Technologies", held at the National University of Comahue, October 2010. Contents: Digital design using VHDL for synthesis and simulation.}
\cvitem{2009}{Seminar "Introduction to Programmable Logic Technologies", held at the National University of Catamarca, on August 12-14, 2009, organized by the IEEE student branch of that university. Contents: Digital design using VHDL for synthesis and simulation.}
\cvitem{2011}{Laboratory Teaching Assistant in the course "ICTP Latin-American Basic Course on FPGA Design for Scientific Instrumentation", organized by the International Centre for Theoretical Physics (ICTP), in Mar del Plata, Buenos Aires, Argentina. February 2011.}

\section{Awards and Distinctions}

\cvitem{2009}{Third place in the "IB50K" business plan competition. Balseiro Institute.}
\cvitem{2008}{Flag bearer of the province in 2008.}


\section{Informatics and Programming Tools}

\subsection{Operating Systems \& Tools}
\cvitem{}{Windows in all its versions}
\cvitem{}{Linux: Debian, Ubuntu}

\subsection{Office Suite}
\cvitem{}{Word, Excel, PowerPoint}
\cvitem{}{LibreOffice}
\cvitem{}{LaTeX (TeXnicCenter, TexMaker)}
\cvitem{}{Full G-Suite (Docs, Sheets, Presentations, Draw, Drive, Gmail)}

\subsection{Image Processing}
\cvitem{}{GIMP}
\cvitem{}{ImageJ}
\cvitem{}{Adobe Lightroom}


\subsection{Programming Languages, HDL, Scripting \& Tools}
\cvitem{}{C, Matlab (Advanced)}
\cvitem{}{VHDL \& Verilog (Advanced)}
\cvitem{}{Python (Advanced)}
\cvitem{}{Makefile}
\cvitem{}{TCL}
\cvitem{}{Advanced knowledge in GIT, Gitlab and Gitlab Continuous Integration.}
\cvitem{}{Docker Containers, Images and Compose.}
\cvitem{}{Databases and Data Visualization: InfluxDB, ElasticSearch, Grafana.}


\subsection{Electronics \& Simulation Tools}
\cvitem{}{Altium Designer}
\cvitem{}{KiCAD}
\cvitem{}{AMD Vitis \& Vivado}
\cvitem{}{Intel Quartus}
\cvitem{}{Microchip Libero}


\section{Experience in Leadership, Mentoring, and NGOs}

\subsection{PMI Nuevo Cuyo / PMI Latin America}

\cvitem{2023}{Mentor in the Leaders of the Present and Future Program, 2023 edition.}

\subsection{Mentorship Program}

\cvitem{2021}{Mentor in the Mentorship Program organized by the National Directorate for Strengthening Entrepreneurial Capacities, Small and Medium Enterprises and Entrepreneurs Secretariat, Ministry of Productive Development of the Nation.}

\subsection{The Mars Society Argentina}

\cvitem{2022}{Organizing Committee for NASA Space Apps, Mendoza headquarters.}

\subsection{IEEE Young Professionals Program}

\cvitem{2016}{President of the IEEE Young Professionals Program, Argentina Section.}
\cvitem{2015}{Vice President of the IEEE Young Professionals Program, Argentina Section.}
\cvitem{2014}{Secretary of the IEEE Young Professionals Program, Argentina Section.}

\subsection{IEEE Student Branch}

\cvitem{2011-2012}{Member of the IEEE Student Branch at the National University of Río Cuarto.}
\cvitem{2007}{Former President 2007, IEEE Student Branch, UNSL}
\cvitem{2006}{President 2006, IEEE Student Branch, UNSL. Organizer of the "National Meeting of IEEE Student Branches 2006." City of San Luis, November 16-18, 2006.}
\cvitem{2005}{Coordinator of the Website Commission of the IEEE Student Branch, San Luis.}

%\cvitemwithcomment{English}{Mothertongue}{}
%\cvitemwithcomment{Spanish}{Intermediate}{Conversationally fluent}
%\cvitemwithcomment{Dutch}{Basic}{Basic words and phrases only}

\renewcommand{\listitemsymbol}{-~} % Changes the symbol used for lists

%\cvlistdoubleitem{Piano}{Chess}
%\cvlistdoubleitem{Cooking}{Dancing}
%\cvlistitem{Running}

%----------------------------------------------------------------------------------------

\end{document}
