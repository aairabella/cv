%%%%%%%%%%%%%%%%%%%%%%%%%%%%%%%%%%%%%%%%%
% "ModernCV" CV and Cover Letter
% LaTeX Template
% Version 1.3 (29/10/16)
%
% This template has been downloaded from:
% http://www.LaTeXTemplates.com
%
% Original author:
% Xavier Danaux (xdanaux@gmail.com) with modifications by:
% Vel (vel@latextemplates.com)
%
% License:
% CC BY-NC-SA 3.0 (http://creativecommons.org/licenses/by-nc-sa/3.0/)
%
% Important note:
% This template requires the moderncv.cls and .sty files to be in the same 
% directory as this .tex file. These files provide the resume style and themes 
% used for structuring the document.
%
%%%%%%%%%%%%%%%%%%%%%%%%%%%%%%%%%%%%%%%%%

%----------------------------------------------------------------------------------------
%	PACKAGES AND OTHER DOCUMENT CONFIGURATIONS
%----------------------------------------------------------------------------------------

\documentclass[11pt,a4paper,sans]{moderncv} % Font sizes: 10, 11, or 12; paper sizes: a4paper, letterpaper, a5paper, legalpaper, executivepaper or landscape; font families: sans or roman

\moderncvstyle{casual} % CV theme - options include: 'casual' (default), 'classic', 'oldstyle' and 'banking'
\moderncvcolor{blue} % CV color - options include: 'blue' (default), 'orange', 'green', 'red', 'purple', 'grey' and 'black'

\usepackage{lipsum} % Used for inserting dummy 'Lorem ipsum' text into the template

\usepackage[scale=0.75]{geometry} % Reduce document margins

\hyphenation{miniMIPS}

%\setlength{\hintscolumnwidth}{3cm} % Uncomment to change the width of the dates column
\setlength{\makecvtitlenamewidth}{10cm} % For the 'classic' style, uncomment to adjust the width of the space allocated to your name

%\usepackage[spanish]{babel}
%\usepackage[utf8]{inputenc}

%----------------------------------------------------------------------------------------
%	NAME AND CONTACT INFORMATION SECTION
%----------------------------------------------------------------------------------------

\firstname{Andr\'es Miguel} % Your first name
\familyname{Airabella} % Your last name

% All information in this block is optional, comment out any lines you don't need
\title{Curriculum Vitae}
\address{Barrio Barrancas Coloradas. M479 C9}{San Luis, Argentina}
\mobile{+54 9 266 4362456}
%\phone{(000) 111 1112}
%\fax{(000) 111 1113}
\email{a.airabella@gmail.com}
\homepage{http://about.me/andresmiguel}{about.me/andresmiguel} % The first argument is the url for the clickable link, the second argument is the url displayed in the template - this allows special characters to be displayed such as the tilde in this example
%\extrainfo{additional information}
\photo[70pt][0.4pt]{pictures/picture} % The first bracket is the picture height, the second is the thickness of the frame around the picture (0pt for no frame)
\quote{"Weniger, aber besser" - Dieter Rams}



%----------------------------------------------------------------------------------------

\begin{document}

%----------------------------------------------------------------------------------------
%	COVER LETTER
%----------------------------------------------------------------------------------------

% To remove the cover letter, comment out this entire block

%\clearpage

%\recipient{HR Department}{Corporation\\123 Pleasant Lane\\12345 City, State} % Letter recipient
%\date{\today} % Letter date
%\opening{Dear Sir or Madam,} % Opening greeting
%\closing{Sincerely yours,} % Closing phrase
%\enclosure[Attached]{curriculum vit\ae{}} % List of enclosed documents
%
%\makelettertitle % Print letter title
%
%\lipsum[1-2] % Dummy text
%\lipsum[4] % Dummy text
%
%\makeletterclosing % Print letter signature
%
%\newpage

%----------------------------------------------------------------------------------------
%	CURRICULUM VITAE
%----------------------------------------------------------------------------------------

\makecvtitle % Print the CV title

%----------------------------------------------------------------------------------------
%	EDUCATION SECTION
%----------------------------------------------------------------------------------------

\section{Datos Personales}

\cvitem{Nombre Completo}{Andr\'es Miguel Airabella}
\cvitem{Fecha de nacimiento}{04 de Junio de 1984}
\cvitem{Lugar de nacimiento}{San Rafael, Mendoza, Argentina}
\cvitem{Nacionalidad}{Argentina}
\cvitem{D.N.I.}{30.965.165}
\cvitem{CUIL}{20-30.965.165-1}
\cvitem{Estado civil}{Casado. Tres Hijos.}
\cvitem{Domicilio}{Barrio Barrancas Coloradas, Manzana 479, Casa 9. 5700 - San Luis. San Luis - Argentina}

\cvitem{Tel\'efono}{+54-9-266-4362456}
\cvitem{Email}{a.airabella@gmail.com}


\section{Formaci\'on}

\subsection{Estudios}

\cventry{2011--(Incompleto)}{Doctorado en Ciencias de la Ingenier\'ia}{Universidad Nacional de R\'io Cuarto}{Facultad de Ingenier\'ia}{Tema de Tesis: 	Convertidores CC-CC tolerantes a fallas para aplicaciones en sistemas el\'ectricos h\'ibridos}{Doctorado acreditado por la CONEAU resoluci\'on 023/07}

\cventry{2003--2009}{Ingenier\'ia Electr\'onica con Orientaci\'on en Sistemas Digitales (OSD)}{Universidad Nacional de San Luis}{Facultad de Ciencias F\'isico Matem\'aticas y Naturales}{Promedio: 8.83 (ocho con ochenta y tres), con y sin aplazos}{}

\cventry{1997--2001}{Perito Administrativo Contable con orientaci\'on en inform\'atica}{Colegio San Rafael, 28 P.S. Hermanos Maristas}{San Rafael. Mendoza}{}{}

\subsection{Idiomas}

\cvitem{Ingl\'es}{Nivel 4 cursado y aprobado en escuela de idiomas I.A.I. San Rafael, Mendoza. Ingl\'es I y II, correspondientes a la carrera de Ing. Electr\'onica con OSD. Excelente habla, lectura, interpretaci\'on de textos y escritura}
\cvitem{Italiano}{Nivel A2 cursado en escuela Danthe Alighieri Villa Mercedes, San Luis}
\cvitem{Castellano}{Idioma nativo}

\subsection{Cursos de Posgrado}

\cvitem{2012}{MODELADO Y CONTROL DE M\'AQUINAS EL\'ECTRICAS (Ding-14). Dictado por la Facultad de Ingenier\'ia, Universidad Nacional de R\'io Cuarto. Docentes Dr. Ing. Guillermo O. Garc\'ia y Dr. Ing. Pablo M. De la Barrera Duraci\'on 60 horas. Fecha: 01 de Abril de 2012 a 10 de Julio de 2012. Certificado: 27 de Mayo de 2013. }
\cvitem{2012}{CONVERTIDORES CONMUTADOS (Cod. 2467). Organizado por el Instituto en Investigaciones en Ingenier\'ia El\'ectrica, Universidad Nacional de Sur. Docentes Responsables: Dres. A. Oliva - M. D'Amico. Duraci\'on 80 horas.  Aprobado el 28 de Junio de 2012. Certificado: 30 de Octubre de 2012. }
\cvitem{2010}{ELECTR\'ONICA DE POTENCIA (Dinge-06). Dictado por la Facultad de Ingenier\'ia, Universidad Nacional de R\'io Cuarto. Docentes Dr. Ing. Guillermo O. Garc\'ia y Dr. Ing. Germ\'an Oggier Duraci\'on 60 horas. Fecha: 01 de Septiembre de 2010 a 17 de Diciembre de 2010. Certificado: 20 de Marzo de 2012.}
\cvitem{2012}{CONTROL LINEAL AVANZADO. Dictado por la Facultad de Ingenier\'ia, Universidad Nacional de R\'io Cuarto. Docentes Dr. Ing. Cristian de \'Angelo, asistente: Mg. Lic. Laura P\'erez. Duraci\'on 60 horas. Fecha: 11 de Septiembre de 2012 a 11 de Diciembre de 2012. Certificado: 27 de Mayo de 2013.}
\cvitem{2012}{TEOR\'IA DE LA POTENCIA INSTANT\'ANEA Y APLICACIONES. Universidad Nacional de R\'io Cuarto. Segundo semestre 2012. Responsables: Dr. Ing. G. O. Garc\'ia y Dr. Ing. G.G. Oggier. Duraci\'on 60 horas. Aprobado. 03 de Noviembre de 2014.}
\cvitem{2013}{ELECTR\'ONICA DE POTENCIA EN MICRORREDES CON ALTA PENETRACI\'ON DE ENERG\'IA RENOVABLE. (Ding-26) Universidad Nacional de R\'io Cuarto. 3 de Septiembre al 30 de Diciembre de 2013. Responsables: Dr. Ing. Denizar C. Martins y Dr. Ing. Guillermo. O. Garc\'ia. Duraci\'on 60 horas. Aprobado. Certificado: 29 de Diciembre de 2014. }
\cvitem{2015}{LABORATORIO DE IMPLEMENTACI\'ON DE PROTOTIPOS EXPERIMENTALES. Universidad Nacional de R\'io Cuarto. Responsables: Dr. Ing. Germ\'an G. Oggier y Dr. Ing. Guillermo. O. Garc\'ia. Duraci\'on 80 horas. Aprobado. 2015. }

\subsection{Seminarios de Posgrado}

\cvitem{2012}{SEMINARIO DE INVESTIGACI\'ON EN CONTROL Y CONVERSI\'ON DE LA ENERG\'IA. Organizado por la Facultad de Ingenier\'ia, Universidad Nacional de R\'io Cuarto. Duraci\'on 40 horas. Fecha: 25 de Abril de 2012 a10 de Diciembre de 2012. Certificado: 23 de Mayo de 2013. }
\cvitem{2011}{SEMINARIO DE INVESTIGACI\'ON EN CIENCIAS DE LA INGENIER\'IA. Organizado por la Facultad de Ingenier\'ia, Universidad Nacional de R\'io Cuarto. Duraci\'on 40 horas. Fecha: 06 de Abril de 2011 a06 de Septiembre de 2011. Certificado: 20 de Noviembre de 2012. }
\cvitem{2011}{SEMINARIO DE INVESTIGACI\'ON EN CONTROL Y CONVERSI\'ON DE LA ENERG\'IA. Organizado por la Facultad de Ingenier\'ia, Universidad Nacional de R\'io Cuarto. Fecha23 de Marzo de 2011a 18 de Diciembre de 2011. Duraci\'on 40 horas. Certificado: 11 de Mayo de 2012. }


\subsection{Cursos de capacitaci\'on y perfeccionamiento}

\cvitem{2018}{Curso: ``Dise\~no Avanzado de sistema embebidos en l\'ogica programable: Zynq APSoC, Vivado HLS y SDSoC''. Cr\'edito Horario: 60hs.  LEIS. Electra Training. 19 al 23 de Marzo de 2018. Universidad Nacional de San Luis. San Luis, Argentina. }
\cvitem{2017}{Curso: ``S\'intesis de Alto Nivel para FPGAs con Vivado-HLS''. LEIS. Electra Training. 11 y 12 de Abril de 2017. LEIS. Universidad Nacional de San Luis. San Luis, Argentina. }
\cvitem{2017}{Curso: Accelerating FPGA Desing (FPGA Competence). Bitbis AS (Norway).  8 y 9 de Mayo de 2017, Satellogic S.A. Buenos Aires, Argentina. }
\cvitem{2017}{Curso: Advanced VHDL Verification - Made Simple (FPGA Competence). Bitbis  AS (Norway). 9 al 11 de Mayo de 2017, Satellogic S.A. Buenos Aires, Argentina. }
\cvitem{2009}{Curso ``Advanced Training Course on FPGA Design and VHDL for Hardware Simulation and Synthesis'' en el Centro Internacional de F\'isica Te\'orica (ICTP), en Trieste, Italia. Duraci\'on 120 hs reloj. Octubre, 2009}


\section{Antecedentes laborales en docencia}

\subsection{Categorizaci\'on Docente Investigador}

\cvitem{}{Docente - Investigador ``Categor\'ia V (cinco)''.}

\subsection{Docencia de Grado}

\cvitem{2019--presente}{Profesor Adjunto, Dedicaci\'on Simple, Car\'acter Efectivo. Facultad de Ciencias F\'isico Matem\'aticas y Naturales. Universidad Nacional de San Luis. Fecha: 05 de Febrero de 2019 al presente. Responsable de materias: Arquitectura de Computadoras (Ing. Electr\'onica con Orientaci\'on en Sistemas Digitales), Audio y Video (Tecnicatura en Telecomunicaciones) y Producci\'on Multimedial (Tecnicatura en Telecomunicaciones).}
\cvitem{2017--2019}{Profesor Adjunto, Dedicaci\'on Simple, Car\'acter Interino. Facultad de Ciencias F\'isico Matem\'aticas y Naturales. Universidad Nacional de San Luis. Fecha: 01 de Septiembre de 2017 al 04 de Febrero de 2019. Responsable de materias: Arquitectura de Computadoras (Ing. Electr\'onica con Orientaci\'on en Sistemas Digitales), Audio y Video (Tecnicatura en Telecomunicaciones) y Producci\'on Multimedial (Tecnicatura en Telecomunicaciones).}
\cvitem{2016--2017}{Profesor Adjunto, Dedicaci\'on Exclusiva, Car\'acter Interino. Facultad de Ciencias F\'isico Matem\'aticas y Naturales. Universidad Nacional de San Luis. Fecha: 14 de Abril de 2016 al 30 de Agosto de 2017. Responsable de materias: Arquitectura de Computadoras (Ing. Electr\'onica con Orientaci\'on en Sistemas Digitales), Audio y Video (Tecnicatura en Telecomunicaciones) y Producci\'on Multimedial (Tecnicatura en Telecomunicaciones).}
\cvitem{2013--2016}{Profesor Adjunto, Dedicaci\'on Simple, Car\'acter Temporario. Facultad de Ciencias F\'isico Matem\'aticas y Naturales. Universidad Nacional de San Luis. Fecha: 22 de Julio de 2013 al 13 de Abril de 2016. Responsable de materias: Arquitectura de Computadoras (Ing. Electr\'onica con Orientaci\'on en Sistemas Digitales), Audio y Video (Tecnicatura en Telecomunicaciones) y Producci\'on Multimedial (Tecnicatura en Telecomunicaciones).}
\cvitem{2012--presente}{Ayudant\'ia de Primera, Dedicaci\'on Simple, Car\'acter Efectivo. Facultad de Ciencias F\'isico Matem\'aticas y Naturales. Universidad Nacional de San Luis. Fecha: 01 de Noviembre de 2012 al presente. En licencia sin goce de haberes por incompatibilidad. Responsable de materias: Arquitectura de Computadoras (Ing. Electr\'onica con Orientaci\'on en Sistemas Digitales), Audio y Video (Tecnicatura en Telecomunicaciones) y Producci\'on Multimedial (Tecnicatura en Telecomunicaciones).}
\cvitem{2011--2012}{Ayudant\'ia de Primera, Dedicaci\'on Simple, Car\'acter Temporario. Facultad de Ciencias F\'isico Matem\'aticas y Naturales. Universidad Nacional de San Luis. Fecha: 1 de Abril de 2011 al 30 de Octubre de 2012. Responsable de Trabajos Pr\'acticos: Arquitectura de Computadoras (Ing. Electr\'onica con Orientaci\'on en Sistemas Digitales) y Dise\~no de Sistemas Digitales (Ing. Electr\'onica con Orientaci\'on en Sistemas Digitales). }
\cvitem{2010--2011}{Ayudant\'ia de Primera, Dedicaci\'on Exclusiva, Car\'acter Temporario. Facultad de Ciencias F\'isico Matem\'aticas y Naturales. Universidad Nacional de San Luis. Fecha: 12 de Marzo de 2010 al 31 de Marzo de 2011. Responsable de Trabajos Pr\'acticos: Arquitectura de Computadoras (Ing. Electr\'onica con Orientaci\'on en Sistemas Digitales) y Dise\~no de Sistemas Digitales (Ing. Electr\'onica con Orientaci\'on en Sistemas Digitales).}
\cvitem{2009-2010}{Ayudant\'ia de Segunda, Dedicaci\'on Simple, Car\'acter Temporario. Facultad de Ciencias F\'isico Matem\'aticas y Naturales. Universidad Nacional de San Luis. Fecha: 09 de Junio de 2009 al 11 de Marzo de 2010. Ayudante de Trabajos Pr\'acticos: Arquitectura de Computadoras (Ing. Electr\'onica con Orientaci\'on en Sistemas Digitales) y Dise\~no de Sistemas Digitales (Ing. Electr\'onica con Orientaci\'on en Sistemas Digitales).}

\subsection{Docencia de Posgrado}

\cvitem{2013}{ELECTR\'ONICA DE POTENCIA. Dictado por la Facultad de Ciencias F\'isico Matem\'aticas y Naturales, Universidad Nacional de San Luis. Docentes Dr. Ing. Germ\'an Oggier. Auxiliar: Ing. Andr\'es Miguel Airabella. Responsable: Ing. Cristian Ariel Falco.  Duraci\'on 60 horas. Fecha: 01 de Noviembre de 2013 a 30 de Noviembre de 2013.}

\subsection{Evaluaci\'on de Trabajos Finales de Grado}

\cvitem{2023}{``Crono TDC: Diseño e implementación de un Time to Digital Converter en FPGA''. Autor:Juli\'an Rodriguez. Director Ing. Nicol\'as Alvarez. Co-Director Dr. Federico Izraelevitch. Carrera: Ingenier\'ia Electr\'onica. Escuela de Ciencia y Tecnolog\'ia. UNSAM. 31 de Mayo de 2023.}
\cvitem{2022}{``Detector de obst\'aculos en veredas para asistencia de personas con ceguera.”''. Autor: Carranza, Lucas. Director: Dr. Emanuel, Trabes. Carrera: Ingenier\'ia Electr\'onica con Orientaci\'on en  Sistemas Digitales. UNSL. 4 de Noviembre de 2022.}
\cvitem{2022}{``Detecci\'on y clasificaci\'on de grietas en pavimentos asf\'alticos de carreteras con Visi\'on Artificial''. Autor: Cortez Médici, Emanuel Alfredo. Director: Ing. Ricardo Petrino. Carrera: Ingenier\'ia Electr\'onica con Orientaci\'on en  Sistemas Digitales. UNSL. 1 de Diciembre de 2022.}
\cvitem{2016}{``Redes inal\'ambricas para el desarrollo de comunidades rurales digitales en el distrito de Yauya''. Autor: Ada Luz Caballero Sifuentes. Director: Ing. Alfredo Debattista. Carrera: Ingenier\'ia Electr\'onica con Orientaci\'on en  Sistemas Digitales. UNSL. Diciembre 2016. }
\cvitem{2016}{``Sistema de medici\'on con celda de carga, con excitaci\'on de Corriente Alterna, basado en amplificador tipo lock-in''. Autor: Laura Beatriz Adaro. Director: Carlos Federico Sosa P\'aez. Carrera: Ingenier\'ia Electr\'onica con Orientaci\'on en  Sistemas Digitales. UNSL. Diciembre 2016. }
\cvitem{2011}{``Procesamiento de se\~nales ac\'usticas utilizando l\'ogica programable''. Autor: Felix Leonardo Garro Mart\'inez. Director: Diego Esteban Costa. Carrera: Ingenier\'ia Electr\'onica con Orientaci\'on en  Sistemas Digitales. UNSL. 16 de Septiembre 2011.}
\cvitem{2011}{``Implementaci\'on de funciones b\'asicas del amplificador Lock-in en FPGA''. Autor: Mart\'inez Guevara, Layla Mar\'ia. Director: Ing. Pel\'aez, Esteban Maximiliano. Carrera: Ingenier\'ia Electr\'onica con Orientaci\'on en  Sistemas Digitales. UNSL. 19 de Diciembre de 2011.}

\subsection{Integrante de tribunales examinadores en concursos docentes}

\cvitem{2018}{Titular en Tribunal para concurso de cargo de AUXILIAR DE, PRIMERA CATEGOR\'IA, dedicaci\'on SIMPLE, car\'acter SUPLENTE con destino al \'AREA DE ELECTR\'ONICA Y MICROPROCESADORES del Departamento de F\'isica, Universidad Nacional de San Luis. 28 de marzo de 2018.}
\cvitem{2014}{Titular en Tribunal para concurso de cargo de AUXILIAR DE, PRIMERA CATEGOR\'IA, dedicaci\'on SIMPLE, car\'acter INTERINO con destino al \'AREA DE ELECTR\'ONICA Y MICROPROCESADORES del Departamento de F\'isica, Universidad Nacional de San Luis. 01 de septiembre de 2014.}

\subsection{Direcci\'on y Co-Direcci\'on de Becas y Trabajos Finales}

\cvitem{2017-2018}{Proyecto Final de Carrera: Alumno: Emiliano \'Alvarez. T\'itulo del Plan de Trabajo: Dise\~no de un Banco de Ensayos para Circuitos de Activaci\'on de Transistores de Potencia. A\~no 2016-2017. Finalizado. Rol: Director. }
\cvitem{2016--2017}{Proyecto Final de Carrera: Alumno: Rodrigo Agust\'in Perna. T\'itulo del Plan de Trabajo: Sistema de supervisi\'on y control en FPGA para un cargador solar de bater\'ias. A\~no 2016-2017. Finalizado. Rol: Director. }
\cvitem{2016-2017}{Beca de Investigaci\'on del CIN (Consejo Interuniversitario Nacional): Alumno: Rodrigo Agust\'in Perna. T\'itulo del Plan de Trabajo: Sistema de supervisi\'on y control en FPGA para un cargador solar de bater\'ias. A\~no 2016-2017. Finalizado. Rol: Director. }
\cvitem{2016--presente}{Proyecto Final de Carrera: Alumno: Rodrigo S\'anchez. T\'itulo del Plan de Trabajo: Descripci\'on en HDL de un detector de \'angulo y secuencia positiva en sistemas el\'ectricos trif\'asicos. A\~no 2016-2017. En ejecuci\'on. Rol: Director. }
\cvitem{2012--2014}{Beca TIC 2012 y Proyecto Final de Carrera: Alumno: Mar\'ia Julia Xacur. T\'itulo del Plan de Trabajo: Desarrollo de instrumental espec\'ifico para medir fuerza de ruptura de s\'olidos. A\~no 2012-2014. Finalizado. }
\cvitem{2012--2013}{Beca TIC 2012: Alumno: Gerardo Galo. T\'itulo del Plan de Trabajo: Dise\~no de una fuente para iluminaci\'on a LED robusta, de baja potencia. A\~no 2012-2013. Finalizado. }
\cvitem{2012--2013}{Co-Direcci\'on de Trabajo Final de Carrera, Alumno: Enzo Belpoliti. T\'itulo del Plan de Trabajo: Implementaci\'on de la T\'ecnica de Modulaci\'on SVPWM para Control de Inversores Trif\'asicos.  A\~no 2012-2013. Finalizado. }

\subsection{Direcci\'on y Co-Direcci\'on de Estancias de Investigaci\'on}
\cvitem{2012}{Direcci\'on de Beca ``Verano de la Ciencia, Universidad Aut\'onoma San Luis Potos\'i 2012''. Alumno: Oscar Zamarripa. T\'itulo del Trabajo: ``Operaci\'on de un Convertidor CC-CC con Puentes Duales Activos en modo Forward para el estudio de fallas en semiconductores''. Lugar: ``Grupo de Electr\'onica Aplicada - Facultad de Ingenier\'ia- Universidad Nacional de R\'io Cuarto. Becario de Intercambio M\'exico-Argentina. Mayo a Junio de 2012.}
\cvitem{2013--2014}{Direcci\'on de Beca de Intercambio ``UNSL-Hochschule Bonn Rhein Sieg''. Alumno: Sven Stockhausen. T\'itulo del Trabajo: ``Dise\~no de un instrumento para medir la fuerza de ruptura de s\'olidos''. Lugar: Laboratorio de Electr\'onica, Investigaci\'on y Servicios - Universidad Nacional de San Luis. Becario de Intercambio Alemania-Argentina. Octubre de 2013 a Febrero de 2014.}

\section{Antecedentes laborales en el sector privado}

\cvitem{2021 al presente}{Enero 2021 al presente: Satellogic S.A. Payload System Flight Engineer. En este puesto, me encargo de la puesta en marcha, calibración y mantenimiento de los sensores de imágenes de una flota de más de 35 satélites en órbita.}
\cvitem{2017--2020}{Mayo de 2017 a Diciembre 2020: Satellogic S.A. Ingeniero de Desarrollo en HDL (Hardware Description Language). En este puesto participé en el diseño de los payloads de satélites de observación terrestre. }
\cvitem{2016--2017}{Agosto 2016 a Abril de 2017: Asesor del Gobierno de la Provincia de San Luis en Energ\'ias Renovables y Eficiencia Energ\'etica. Programa de Energ\'ias Renovables y Eficiencia Energ\'etica, Ministerio de Medio Ambiente, Campo y Producci\'on. }
\cvitem{2009--2010}{Junio de 2009 a Abril de 2010: Becario del Grupo de Estudios Ambientales, para soporte en desarrollo de equipos electr\'onicos de medici\'on de variables ambientales.}
\cvitem{2008--2009}{Octubre 2008 a Marzo de 2009: Dise\~no de m\'odulos/macros digitales (IP Cores) optimizados para bajo consumo de potencia en lenguajes de descripci\'on de hardware para la empresa MGB Design SRL, a su vez para Actel Corp (Ahora Microsemi).}


\section{Antecedentes en Investigaci\'on Cient\'ifica y Desarrollo Tecnol\'ogico}

\subsection{Participaci\'on en Programas y Proyectos de Investigaci\'on y Desarrollo}
\cvitem{2016}{Proyecto de extensi\'on universitaria "F\'abricas Recuperadas por los Trabajadores". aceptado mediante la resoluci\'on CS 279-16 Universidad Nacional de San Luis. Funci\'on: Integrante.}
\cvitem{2015--2016}{``Equipo para medir dureza''. Proyecto del Programa Universidad, Dise\~no y Desarrollo Productivo 2015. Funci\'on: Director del proyecto. (Subsidio otorgador: \$25.000,00)}
\cvitem{2015--2016}{``Medidor Inteligente de Energ\'ia''. Proyecto del Programa Universidad, Dise\~no y Desarrollo Productivo 2015. Funci\'on: Director del proyecto. (Subsidio otorgador: \$25.000,00)}
\cvitem{2015--2017}{(Proyecto bianual) PDTS-CIN-CONICET PDTS209. ``Sistema modular de tracci\'on para veh\'iculos el\'ectricos''. Inv. Responsable: Cristian De Angelo,  \'Area: Ingenier\'ia y Tecnolog\'ia. Universidad Nacional de R\'io Cuarto. Subsidio otorgado \$199.800. El proyecto incluye una beca posdoctoral. CIN-CONICET, Res. CE Nº 1055-15.}
\cvitem{2015--2018}{(Proyecto trianual) PICT-2014-2760. ``Veh\'iculos Urbanos de Tracci\'on El\'ectrica: Sistema de Propulsi\'on y Gesti\'on de Energ\'ia''. Inv. Responsable: Cristian De Angelo,  \'Area: Tecnolog\'ia Energ\'etica Minera Mec\'anica y de Materiales. Universidad Nacional de R\'io Cuarto. Subsidio otorgado \$475.000. El proyecto incluye una beca de doctorado. ANPCyT, Res. Nº 270-15.}
\cvitem{2015--2018}{(Proyecto trianual). CONICET PIP 2014-2016 GI ``Veh\'iculos Urbanos de Tracci\'on El\'ectrica: control, supervisi\'on, gesti\'on de energ\'ia e integraci\'on a la red el\'ectrica''. Director: Garc\'ia, Guillermo O. Co-director: De Angelo, Cristian H. Aprobado. Subsidio otorgado \$494.000. Res. 5013/14. }
\cvitem{2015--2018}{(Proyecto trianual) PICT-Start-Up 2014-3647 ``I+D de Tecnolog\'ias para UPS de Alta Potencia''. Director: Guillermo O. Garc\'ia. Proyecto conjunto entre CREXEL S.A. y el GEA-UNRC. Monto total otorgado \$600.000. }
\cvitem{2014--2015}{(Proyecto Bianual). PROIPRO 142514. ``Control de convertidores de potencia para sistemas de energ\'ias renovables''. Financiado por Universidad Nacional de San Luis. L\'inea de trabajo: ``L\'ogica programable aplicada a la electr\'onica de potencia''. Director del Proyecto: De \'Angelo, Cristian. Director de la L\'inea: Falco, Cristian Ariel. }
\cvitem{2014--2017}{(Proyecto trianual) PICT-2013-1194 ``Paralelismo de inversores trif\'asicos para la integraci\'on de energ\'ias renovables en microrredes'' Temas Abiertos. Garc\'ia, Guillermo Oscar. Tecnolog\'ia Energ\'etica, Minera, Mec\'anica y de Materiales, Universidad Nacional de R\'io Cuarto, Subsidio total + gastos de administraci\'on \$ 504.000.}
\cvitem{2014--2015}{``Banco de ensayo para drivers''. Proyecto del Programa Universidad, Dise\~no y Desarrollo Productivo 2014. Funci\'on: Director del proyecto. (Subsidio otorgador: \$24.987,00)}
\cvitem{2014--2015}{``PLD en electr\'onica de potencia'' Proyecto del Programa Universidad, Dise\~no y Desarrollo Productivo 2014. Funci\'on: Docente colaborador. (Subsidio otorgador: \$25.000,00)}
\cvitem{2012--2013}{``Energ\'ias alternativas para el suministro de energ\'ia el\'ectrica en la Regi\'on Centro Oeste''. Proyecto ANPCyT FONCyT CIN II PICTO-2010-0154 (subsidio otorgado \$200.000).}
\cvitem{2011--2013}{``Sistemas de Generaci\'on Electro-E\'olicos - Escalas de 5 y 30 kW''. Proyecto conjunto entre el FONCyT, ANPCyT, PICT-2010, \#2744 (subsidio otorgado \$280.000). }
\cvitem{2011--2013}{``Detecci\'on y Diagn\'ostico de Fallas Incipientes en Accionamientos El\'ectricos''. PIP CONICET 2010. Subsidio otorgado \$180.000. }
\cvitem{2013--2014}{``Control del flujo de energ\'ia el\'ectrica de cc tolerante a fallas''. Proyecto subsidiado por el Ministerio de Ciencia y Tecnolog\'ia de la Provincia de C\'ordoba. Res. Nº 113/2011. Monto a otorgar \$40.000. Director: Dr. Ing. Germ\'an Gustavo Oggier.}
\cvitem{2012--2015}{``Programa Control y Conversi\'on de Energ\'ia''. Programa de Investigaci\'on subsidiado por la Secretaria de Ciencia y T\'ecnica de la UNRC. Res. Rec. 328/12 (\$71.456). El Programa contiene a los siguientes Proyectos: Sistemas de Generaci\'on Electro-E\'olicos, Veh\'iculos El\'ectricos e H\'ibridos, Diagn\'ostico de Fallas Incipientes en Accionamientos El\'ectricos.}
\cvitem{2012}{Proyecto de Cooperaci\'on Internacional ``Topolog\'ias Electr\'onicas de Potencia para una Integraci\'on Eficiente de las Fuentes de Energ\'ias Renovables en Micro-Redes y Sistemas El\'ectricos'', subsidiado por el Ministerio de Ciencia y Tecnolog\'ia de la Provincia de C\'ordoba. Proyecto conjunto entre el GEA-UNRC, Argentina y INEP-UFSC, Brasil. Proyecto CCII -11347549-A-2011. Res. 174/11. }
\cvitem{2009--2011}{``Programa Control y Conversi\'on de Energ\'ia''. Programa de Investigaci\'on subsidiado por la Secretaria de Ciencia y T\'ecnica de la UNRC. Res. Rec. 442/09 (\$18.000). El Programa contiene a los siguientes Proyectos: Sistemas de Generaci\'on Electro-E\'olicos, Veh\'iculos El\'ectricos e H\'ibridos, Diagn\'ostico de Fallas Incipientes en Accionamientos El\'ectricos.}

\subsection{Becas Obtenidas}
\cvitem{2014--2015}{Beca Interna de Formaci\'on de Posgrado Tipo II. Proyecto: ``Convertidores CC-CC tolerantes a fallas para aplicaciones en sistemas el\'ectricos h\'ibridos''. Otorgado por el Consejo Nacional de Investigaciones Cient\'ificas y T\'ecnicas (CONICET). Fecha: Abril de 2014 a Marzo de 2016. }
\cvitem{2011--2013}{Beca Interna de Formaci\'on de Posgrado Tipo I. Proyecto: ``Convertidores CC-CC tolerantes a fallas para aplicaciones en sistemas el\'ectricos h\'ibridos''. Otorgado por el Consejo Nacional de Investigaciones Cient\'ificas y T\'ecnicas (CONICET). Fecha: Abril de 2011 a Marzo de 2014. }
\cvitem{2009}{International Center for Theoretical Physics (ICTP). Full Schoolarship para el curso ``Advanced Training Course on FPGA Design and VHDL for Hardware Simulation and Synthesis''. Trieste, Italy. October and November 2009. }
\cvitem{2007--2008}{Beca Est\'imulo de la Facultad de Ciencias F\'isico Matem\'aticas y Naturales. Trabajo de Investigaci\'on (Trabajo Final de Carrera): ``Procesamiento de Im\'agenes en FPGA''.}

\section{Producci\'on en Investigaci\'on Cient\'ifica - Publicaciones}


\subsection{Trabajos publicados en revistas internacionales indexadas}

\cvitem{2015}{\textbf{A.M. Airabella}, G.G. Oggier, L.E. Piris-Botalla, C. A. Falco, and G.O. Garc\'ia, ``Semiconductors Faults Analysis in Dual Active Bridge DC-DC Converter'' IET Power Electronics. Octubre-2015. DOI:  10.1049/iet-pel.2015.0299 , Online ISSN 1755-4543 }
\cvitem{2014}{L. E. Piris Botalla, G. G. Oggier, \textbf{A. M. Airabella} y G. O. Garc\'ia, ``Power Losses Evaluation of a Bidirectional Three-Port DC-DC Converter for Hybrid Electric System''. Elsevier International Journal of Electrical Power \& Energy Systems (ISSN: 0142-0615). Volume 58, June 2014, Pages 1-8}
\cvitem{2016}{L. E. Piris Botalla, G. G. Oggier, \textbf{A. M. Airabella} y G. O. Garc\'ia, ``Extensi\'on del Rango de Operaci\'on con Conmutaci\'on Suave de un Convertidor CC-CC Bidireccional de Tres Puertos'' Revista Iberoamericana de Autom\'atica e Inform\'atica Industrial (RIAI) (ISSN: 1697-7912) - Aceptado Marzo de 2015.}

%\subsection{Trabajos en prensa en revistas internacionales indexadas}

%\subsection{Trabajos en evaluaci\'on en revistas internacionales indexadas}

\subsection{Trabajos publicados en revistas nacionales no indexadas}

\cvitem{2015}{\textbf{A. M. Airabella}, G. G. Oggier, L. E. Piris-Botalla, Cristian A. Falco y G. O. Garc\'ia, ``Estrategia de Detecci\'on de Fallas de Circuito Abierto en Semiconductores de Convertidores CC-CC Aislados'' Nota T\'ecnica, Revista Ingenier\'ia El\'ectrica. Editores On-Line S.R.L. Marzo 2015. }

%\subsection{Trabajos en evaluaci\'on en Congresos con referato nacional}

\subsection{Trabajos Completos en Congresos con referato internacional}

\cvitem{2023}{\textbf{A. M. Airabella.} ``Ambitious: Edge AI Platform''. XII Congreso Argentino de Tecnología Espacial. 12-14 de Abril, 2023. Mendoza, Argentina.}
\cvitem{2019}{\textbf{A. M. Airabella}, D. Caruso, A. J. Demski. ``BitSync: A novel Data to Clock Phase Alignment for Microsemi FPGAs''. SPL Conf. April 10th to 12th, 2019. Buenos Aires, Argentina.}
\cvitem{2019}{A. J. Demski, D. Caruso, \textbf{A. M. Airabella}. ``Propuesta de solucion de problema de Clock Domain Crossing en un IP Core AXI to AHB''. SPL Conf. April 10th to 12th, 2019. Buenos Aires, Argentina.}
\cvitem{2019}{D. Caruso, \textbf{A. M. Airabella}, R. A. Melo. ``High-Speed serial protocol multi-link and multi-stage for FPGAs''. SPL Conf. April 10th to 12th, 2019. Buenos Aires, Argentina.}
\cvitem{2018}{\textbf{A. M. Airabella}, C. Falco, G. Oggier y G. O. Garcia. ``Detecci\'on, Diagn\'ostico y Tolerancia a Fallas de Transistores en Convertidores CC-CC con Puentes Duales Activos en Modo Reductor''. IEEE ARGENCON 2018. 6, 7 y 8 de Junio de 2018. San Miguel de Tucum\'an}
\cvitem{2012}{\textbf{A. M. Airabella}, G. G. Oggier, L. E. Piris-Botalla, Cristian A. Falco y G. O. Garc\'ia, ``Open Transistors and Diodes Fault Diagnosis Strategy for Dual Active Bridge DC-DC Converter,'' 10th IEEE/IAS International Conference on Industry Applications (INDUSCON 2012). 5th to 7th of November of 2012 in Fortaleza.}
\cvitem{2012}{L.E. Piris-Botalla, G. G. Oggier, \textbf{A. M. Airabella} and G. O. Garc\'ia. ``Analysis and evaluation of power switch losses for three-port bidirectional DC-DC converter'' 2012 IEEE International Conference on Industrial Technology (ICIT),  Athens, Greece at 19-21 March 2012.}

\subsection{Trabajos Completos en Congresos con referato nacional}

\cvitem{2017}{M. E. Postemsky, S. F. Hern\'andez Vel\'azquez, R. M. Murdocca, \textbf{A. M. Airabella}. ``Medidor Inteligente de Energ\'ia''. VIII Congreso de Microelectr\'onica Aplicada. C\'ordoba 11 al 13 de octubre de 2017. }
\cvitem{2016}{R. A. Perna, O. E. \'Alvarez, J. R. S\'anchez y \textbf{A. M. Airabella}. L\'ogica de Control de un Banco de Ensayos para Circuitos de Activaci\'on de Transistores de Potencia. VII Congreso de Microelectr\'onica Aplicada 2016. Universidad Nacional de San Luis, San Luis, Argentina. }
\cvitem{2016}{O. E. Alvarez, R. A. Perna, J. R. S\'anchez y \textbf{A. M. Airabella}. Implementaci\'on de un banco de ensayos para circuitos de activaci\'on de transistores de potencia. VII Congreso de Microelectr\'onica Aplicada 2016. Universidad Nacional de San Luis, San Luis, Argentina. }
\cvitem{2016}{\textbf{A. M. Airabella}, G. G. Oggier, L. E. Piris-Botalla, C. A. Falco, G. O. Garc\'ia, ``L\'imites de Transferencia de Potencia de un Convertidor CC-CC con Puentes Duales Activos Tolerante a Fallas en Modo Reductor'', XVI Reuni\'on de Trabajo en Procesamiento de la Informaci\'on y Control (XVI RPIC) Ciudad de C\'ordoba, 6 al  8 de octubre de 2015. }
\cvitem{2015}{L. E. Piris-Botalla, G. G. Oggier, \textbf{A. M. Airabella} y G. O. Garc\'ia, ``Dise\~no de las Inductancias Auxiliares de un Convertidor CC-CC Bidireccional de Tres Puertos'' XVI Reuni\'on de Trabajo en Procesamiento de la Informaci\'on y Control (XVI RPIC) Ciudad de C\'ordoba, 6 al  8 de octubre de 2015. }
\cvitem{2015}{L. E. Piris-Botalla, G. G. Oggier, \textbf{A. M. Airabella} y G. O. Garc\'ia, ``Comparaci\'on de las P\'erdidas de un Convertidor CC-CC Bidireccional de Tres Puertos para Diferentes Valores de Inductancias Auxiliares'' XVI Reuni\'on de Trabajo en Procesamiento de la Informaci\'on y Control (XVI RPIC) Ciudad de C\'ordoba, 6 al  8 de octubre de 2015.}
\cvitem{2015}{L. E. Piris-Botalla, G. G. Oggier, \textbf{A. M. Airabella} y G. O. Garc\'ia, ``Determinaci\'on y Extensi\'on de los L\'imites de Operaci\'on de un Convertidor CC-CC Bidireccional de Tres Puertos'' XVI Reuni\'on de Trabajo en Procesamiento de la Informaci\'on y Control (XVI RPIC) Ciudad de C\'ordoba, 6 al  8 de octubre de 2015.}
\cvitem{2014}{E. Belpoliti, \textbf{A. M. Airabella}, Cristian Ariel Falco ``T\'ecnica svpwm en fpga para el control de inversores trif\'asicos''. 24\textsuperscript{o} Congreso Argentino de Control Autom\'atico. 27 al 29 de Octubre de 2014 - Buenos Aires, Argentina.}
\cvitem{2013}{\textbf{A. M. Airabella}, G. G. Oggier, L. E. Piris-Botalla, C. A. Falco, G. O. Garc\'ia, ``Esquemas Tolerantes a Fallas de Transistores de Potencia Para Convertidores con Puentes Duales Activos'', XV Reuni\'on de Trabajo en Procesamiento de la Informaci\'on y Control (XV RPIC) S.C. de Bariloche, 16 al  20 de setiembre de 2013.}
\cvitem{2011}{\textbf{A. M. Airabella}, G. G. Oggier, L. E. Piris-Botalla, Cristian A. Falco y G. O. Garc\'ia, ``Estrategia de Detecci\'on de Fallas de Circuito Abierto en Semiconductores de Convertidores CC-CC Aislados'' AADECA 2012, 3 al 5 de Octubre 2012, Buenos Aires, Argentina.}
\cvitem{2011}{A. I. Testa, M. R. Palavecino Nicotra, \textbf{A. M. Airabella}, F. Aguilera. ``Implementaci\'on de Sistema Embebido Sobre FPGA Basado en el Microprocesador miniMIPS''. III Congreso de Micro-Electr\'onica Aplicada 2012. Secci\'on Estudiantil. 2012. ISBN 978-987-702-004-5.}
\cvitem{2011}{\textbf{A. M. Airabella}, G. G. Oggier, L. E. Piris-Botalla y G. O. Garc\'ia, ``Diagn\'ostico de fallas en semiconductores de potencia de Convertidores CC-CC con Puentes Duales Activos'' XIV Reuni\'on en Procesamiento de la Informaci\'on y Control (RPIC 2011), Universidad Nacional de Entre R\'ios, Oro Verde, Entre R\'ios, Argentina. Del 16 al 18 de noviembre de 2011.}
\cvitem{2011}{L. E. Piris-Botalla, G. G. Oggier, \textbf{A. M. Airabella} y G. O. Garc\'ia ``Convertidor CC-CC bidireccional de tres puertos: Evaluaci\'on de las p\'erdidas en las llaves de potencia'', XIV Reuni\'on en Procesamiento de la Informaci\'on y Control (RPIC 2011), Universidad Nacional de Entre R\'ios, Oro Verde, Entre R\'ios, Argentina. Del 16 al 18 de noviembre de 2011.}
\cvitem{2010}{\textbf{A. M. Airabella}, C. Sosa P\'aez, R. Petrino: ``Plataforma para Procesamiento de Im\'agenes en FPGA'', I Congreso de Microelectr\'onica Aplicada, UnLAM 2010. }


\section{Participaci\'on en Reuniones Cient\'ificas y Educativas}

\subsection{Como Autor de Trabajos}

\cvitem{2013}{XVI Reuni\'on de Trabajo en Procesamiento de la Informaci\'on y Control (XVI RPIC). T\'itulo del trabajo: ``L\'imites de Transferencia de Potencia de un Convertidor CC-CC con Puentes Duales Activos Tolerante a Fallas en Modo Reductor'', Ciudad de C\'ordoba, 6 al  8 de octubre de 2015. }
\cvitem{2012}{AADECA’12 - Semana del Control Autom\'atico. 23º Congreso Argentino de Control Autom\'atico. 3 al 5 de Octubre de 2012. Buenos Aires, Argentina. T\'itulo del Trabajo: ``Estrategia de detecci\'on de fallas de circuito abierto en semiconductores de convertidores CC-CC aislados''. }
\cvitem{2011}{XIV Reuni\'on en Procesamiento de la Informaci\'on y Control (RPIC 2011), Universidad Nacional de Entre R\'ios, Oro Verde, Entre R\'ios, Argentina. Del 16 al 18 de noviembre de 2011. T\'itulo del Trabajo: ``Diagn\'ostico de fallas en semiconductores de potencia de Convertidores CC-CC con Puentes Duales Activos''.}
\cvitem{2010}{I Congreso de Microelectr\'onica Aplicada. 2010. La Matanza, Buenos Aires, Argentina. T\'itulo del Trabajo: ``Plataforma para Procesamiento de Im\'agenes en FPGA''.}

\subsection{Como Conferencista, Docente u Organizador}

\cvitem{2018}{``Satellogic: Una charla sobre sat\'elites''. 21 de Marzo de 2018, Universidad Nacional de San Luis. Organizada por Rama Estudiantil IEEE San Luis. }
\cvitem{2017}{SASE 2017 	- Disertante del Workshop: ``Punto de Partida para Dise\~nos FPGA-VHDL'' en el Simposio Argentino de Sistemas Embebidos (SASE 2017), 9 al 11 de Agosto de 2017. }
\cvitem{2016}{uEA 2016 - VII Congreso de Microelectr\'onica Aplicada 2016. Universidad Nacional de San Luis. 26, 27 y 28 de Octubre de 2016, San Luis, Argentina. }
\cvitem{2015}{III JOREIC - 3eras Jornadas Regionales de Estudiantes de Ingenier\'ia Civil 2015. ``Ten\'es dos minutos? C\'omo transmitir ideas en tiempo l\'imite''. Universidad Tecnol\'ogica Nacional. Facultad Regional San Rafael. 22 de Mayo de 2015.}
\cvitem{2014}{XX JOSEII. T\'itulo de la conferencia: ``Ten\'es dos minutos? C\'omo transmitir ideas en tiempo l\'imite''. Universidad Tecnol\'ogica Nacional. Facultad Regional San Rafael. 7 de Noviembre de 2014.}
\cvitem{2014}{T\'itulo de la conferencia: ``Prohibido Buscar Trabajo''. Universidad Nacional de Cuyo, Facultad de Ciencias Aplicadas a la Industria. Organizado por AECA. San Rafael. 4 de Noviembre de 2014.}
\cvitem{2014}{II Semana Nacional del Emprendedor Tecnol\'ogico. T\'itulo de la conferencia: ``Prohibido Buscar Trabajo''. Universidad Tecnol\'ogica Nacional. Facultad Regional San Rafael. 16 de Septiembre de 2014. }
\cvitem{2012}{S\'eptimo Encuentro Nacional de Estudiantes de Ingenier\'ia (ENEI 2012). T\'itulo de la conferencia: ``Ingeniero? Prohibido Buscar Trabajo''. 18 y 19 de Agosto de 2012. San Luis, Argentina. }
\cvitem{2011}{Simposio Argentino de Sistemas Embebidos 2011. Buenos Aires, Argentina. T\'itulo de la conferencia: ``Simulaci\'on Avanzada con TestBench en HDL''.}
\cvitem{2010}{Quinto Encuentro Nacional de Estudiantes de Ingenier\'ia (ENEI 2010). T\'itulo de la conferencia: ``Ingeniero? Prohibido Buscar Trabajo''.25, 26  y 27 de Noviembre de 2010, R\'io Cuarto, C\'ordoba, Argentina. }
\cvitem{2010}{I Congreso de Microelectr\'onica Aplicada. 2010. La Matanza, Buenos Aires, Argentina. T\'itulo de la conferencia: ``Simulaci\'on Avanzada con TestBench en HDL''.}
\cvitem{2010}{Seminario ``Introducci\'on a las Tecnolog\'ias de L\'ogica Programable'', realizado en la Universidad Nacional del Comahue, Octubre 2010. Contenidos: Dise\~no digital usando VHDL para s\'intesis y simulaci\'on.}
\cvitem{2009}{Seminario ``Introducci\'on a las Tecnolog\'ias de L\'ogica Programable'', realizado en la Universidad Nacional de Catamarca, los d\'ias 12, 13 y 14 de agosto de 2009, organizado por la rama estudiantil del IEEE de dicha universidad. Contenidos: Dise\~no digital usando VHDL para s\'intesis y simulaci\'on.}
\cvitem{2011}{Docente-Tutor de Laboratorio en el curso ``ICTP Latin-American Basic Course on FPGA Design for Scientific Instrumentation'', organizado por el Centro Internacional de F\'isica Te\'orica (ICTP), en Mar del Plata, Buenos Aires, Argentina. Febrero 2011. }

\subsection{Como Asistente}

\cvitem{2011}{Ciclo de Conferencia sobre Energ\'ia E\'olica 2011. Organizado por: Grupo de Electr\'onica Aplicada, Grupo de An\'alisis de Sistemas El\'ectricos de Potencia, Cap\'itulo Conjunto Argentino de IEEE y Cap\'itulo Argentino IEEE PES. 16 de Julio de 2011. R\'io Cuarto, C\'ordoba, Argentina. }


\section{Premios y distinciones}

\cvitem{2008}{Abanderado a\~no 2008, portador de bandera provincial. }
\cvitem{2009}{Tercer puesto en concurso de planes de negocios ``IB50K''. Instituto Balseiro.}



\section{Herramias inform\'aticas y de programaci\'on}

\subsection{Sistemas Operativos}
\cvitem{}{Windows en todas sus versiones}
\cvitem{}{Linux: Debian, Ubuntu. }
\subsection{Ofim\'atica}
\cvitem{}{Word, Excel, PowerPoint}
\cvitem{}{LibreOffice}
\cvitem{}{LaTex (TeXnicCenter, TexMaker)}
\subsection{Procesamiento de Im\'agenes}
\cvitem{}{GIMP}
\cvitem{}{Adobe Lightroom}
\subsection{Lenguajes de Programaci\'on, HDL y Scripting}
\cvitem{}{C, Matlab (Avanzado)}
\cvitem{}{VHDL (Avanzado)}
\cvitem{}{python}
\cvitem{}{Verilog}
\cvitem{}{Makefile}
\cvitem{}{TCL}

\section{Experiencia en Liderazgo, Mentoring y ONGs}

\subsection{PMI Nuevo Cuyo / PMI Latinoamérica}

\cvitem{2023}{Mentor en el Programa Líderes del Presente y del Futuro, edición 2023.}

\subsection{Programa de Mentores}

\cvitem{2021}{Mentor en el Programa de Mentores, organizado por la Dirección Nacional de Fortalecimiento de Capacidades Emprendedoras, Secretaría de la Pequeña y la Mediana Empresa y los Emprendedores, Ministerio de Desarrollo Productivo de la Nación.}

\subsection{The Mars Society}

\cvitem{2022}{Comité Organizador NASA Space Apps, sede Mendoza.}


\subsection{IEEE Young Professionals Programm}

\cvitem{2016}{Presidente de IEEE Young Professionals Program de la Secci\'on Argentina.}
\cvitem{2015}{Vicepresidente de la Secci\'on Argentina de IEEE Young Professionals Program.}
\cvitem{2014}{Secretario de la Secci\'on Argentina de IEEE Young Professionals Program.}

\subsection{Rama Estudiantil del IEEE}

\cvitem{2011-2012}{Miembro de la Rama Estudiantil del IEEE de la Universidad Nacional de R\'io Cuarto.}
\cvitem{2007}{Former President 2007. Rama Estudiantil IEEE UNSL}
\cvitem{2006}{Presidente 2006. Rama Estudiantil IEEE UNSL. Organizador ``Reuni\'on Nacional de Ramas Estudiantiles del IEEE 2006''. Ciudad de San Luis, del 16 al 18 de Noviembre de 2006.}
\cvitem{2005}{Coordinador de la Comisi\'on Sitio Web de la Rama Estudiantil IEEE San Luis.}

%\cvitemwithcomment{English}{Mothertongue}{}
%\cvitemwithcomment{Spanish}{Intermediate}{Conversationally fluent}
%\cvitemwithcomment{Dutch}{Basic}{Basic words and phrases only}

\section{Intereses}

\cvitem{Deportes}{Running. Ciclismo. Escalada.}
\cvitem{Hobbies}{Motociclismo y automovilismo. Leer y viajar.}

\renewcommand{\listitemsymbol}{-~} % Changes the symbol used for lists

%\cvlistdoubleitem{Piano}{Chess}
%\cvlistdoubleitem{Cooking}{Dancing}
%\cvlistitem{Running}

%----------------------------------------------------------------------------------------

\end{document}
