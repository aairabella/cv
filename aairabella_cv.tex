%%%%%%%%%%%%%%%%%%%%%%%%%%%%%%%%%%%%%%%%%
% "ModernCV" CV and Cover Letter
% LaTeX Template
% Version 1.3 (29/10/16)
%
% This template has been downloaded from:
% http://www.LaTeXTemplates.com
%
% Original author:
% Xavier Danaux (xdanaux@gmail.com) with modifications by:
% Vel (vel@latextemplates.com)
%
% License:
% CC BY-NC-SA 3.0 (http://creativecommons.org/licenses/by-nc-sa/3.0/)
%
% Important note:
% This template requires the moderncv.cls and .sty files to be in the same 
% directory as this .tex file. These files provide the resume style and themes 
% used for structuring the document.
%
%%%%%%%%%%%%%%%%%%%%%%%%%%%%%%%%%%%%%%%%%

%----------------------------------------------------------------------------------------
%	PACKAGES AND OTHER DOCUMENT CONFIGURATIONS
%----------------------------------------------------------------------------------------

\documentclass[11pt,a4paper,sans]{moderncv} % Font sizes: 10, 11, or 12; paper sizes: a4paper, letterpaper, a5paper, legalpaper, executivepaper or landscape; font families: sans or roman

\moderncvstyle{casual} % CV theme - options include: 'casual' (default), 'classic', 'oldstyle' and 'banking'
\moderncvcolor{blue} % CV color - options include: 'blue' (default), 'orange', 'green', 'red', 'purple', 'grey' and 'black'

\usepackage{lipsum} % Used for inserting dummy 'Lorem ipsum' text into the template

\usepackage[scale=0.75]{geometry} % Reduce document margins
%\setlength{\hintscolumnwidth}{3cm} % Uncomment to change the width of the dates column
\setlength{\makecvtitlenamewidth}{10cm} % For the 'classic' style, uncomment to adjust the width of the space allocated to your name

%\usepackage[spanish]{babel}
%\usepackage[utf8]{inputenc}

%----------------------------------------------------------------------------------------
%	NAME AND CONTACT INFORMATION SECTION
%----------------------------------------------------------------------------------------

\firstname{Andr\'es Miguel} % Your first name
\familyname{Airabella} % Your last name

% All information in this block is optional, comment out any lines you don't need
\title{Curriculum Vitae}
\address{Barrio Barrancas Coloradas. M479 C9}{San Luis, Argentina}
\mobile{+54 9 266 4362456}
%\phone{(000) 111 1112}
%\fax{(000) 111 1113}
\email{a.airabella@gmail.com}
\homepage{http://www.unsl.edu.ar/~amairabe}{www.unsl.edu.ar/~amairabe} % The first argument is the url for the clickable link, the second argument is the url displayed in the template - this allows special characters to be displayed such as the tilde in this example
%\extrainfo{additional information}
\photo[70pt][0.4pt]{pictures/picture} % The first bracket is the picture height, the second is the thickness of the frame around the picture (0pt for no frame)
\quote{"Weniger, aber besser" - Dieter Rams}

%----------------------------------------------------------------------------------------

\begin{document}

%----------------------------------------------------------------------------------------
%	COVER LETTER
%----------------------------------------------------------------------------------------

% To remove the cover letter, comment out this entire block

%\clearpage

%\recipient{HR Department}{Corporation\\123 Pleasant Lane\\12345 City, State} % Letter recipient
%\date{\today} % Letter date
%\opening{Dear Sir or Madam,} % Opening greeting
%\closing{Sincerely yours,} % Closing phrase
%\enclosure[Attached]{curriculum vit\ae{}} % List of enclosed documents
%
%\makelettertitle % Print letter title
%
%\lipsum[1-2] % Dummy text
%\lipsum[4] % Dummy text
%
%\makeletterclosing % Print letter signature
%
%\newpage

%----------------------------------------------------------------------------------------
%	CURRICULUM VITAE
%----------------------------------------------------------------------------------------

\makecvtitle % Print the CV title

%----------------------------------------------------------------------------------------
%	EDUCATION SECTION
%----------------------------------------------------------------------------------------

\section{Datos Personales}

\section{Formación}

\subsection{Estudios}

\cventry{2011--Actualidad}{Doctorado en Ciencias de la Ingenier\'ia}{Universidad Nacional de R\'io Cuarto}{Facultad de Ingeniería}{Tema de Tesis: 	Convertidores CC-CC tolerantes a fallas para aplicaciones en sistemas eléctricos híbridos}{Doctorado acreditado por la CONEAU resolución 023/07}

\cventry{2003--2009}{Ingenier\'ia Electr\'onica con Orientación en Sistemas Digitales (OSD)}{Universidad Nacional de San Luis}{Facultad de Ciencias Físico Matemáticas y Naturales}{Promedio: 8.83 (ocho con ochenta y tres), con y sin aplazos}{}

\cventry{1997--2001}{Perito Administrativo Contable con orientación en informática}{Colegio San Rafael, 28 P.S. Hermanos Maristas}{San Rafael. Mendoza}{}{}

\subsection{Idiomas}

\cvitem{Inglés}{Nivel 4 cursado y aprobado en escuela de idiomas I.A.I. San Rafael, Mendoza. Inglés I y II, correspondientes a la carrera de Ing. Electrónica con OSD. Buen habla, lectura, interpretación de textos y escritura}
\cvitem{Castellano}{Idioma nativo}

\subsection{Cursos de Posgrado}

\cvitem{2012}{MODELADO Y CONTROL DE MÁQUINAS ELÉCTRICAS (Ding-14). Dictado por la Facultad de Ingeniería, Universidad Nacional de Río Cuarto. Docentes Dr. Ing. Guillermo O. García y Dr. Ing. Pablo M. De la Barrera Duración 60 horas. Fecha: 01 de Abril de 2012 a 10 de Julio de 2012. Certificado: 27 de Mayo de 2013. }
\cvitem{2012}{CONVERTIDORES CONMUTADOS (Cod. 2467). Organizado por el Instituto en Investigaciones en Ingeniería Eléctrica, Universidad Nacional de Sur. Docentes Responsables: Dres. A. Oliva – M. D’Amico. Duración 80 horas.  Aprobado el 28 de Junio de 2012. Certificado: 30 de Octubre de 2012. }
\cvitem{2010}{ELECTRÓNICA DE POTENCIA (Dinge-06). Dictado por la Facultad de Ingeniería, Universidad Nacional de Río Cuarto. Docentes Dr. Ing. Guillermo O. García y Dr. Ing. Germán Oggier Duración 60 horas. Fecha: 01 de Septiembre de 2010 a 17 de Diciembre de 2010. Certificado: 20 de Marzo de 2012.}
\cvitem{2012}{CONTROL LINEAL AVANZADO. Dictado por la Facultad de Ingeniería, Universidad Nacional de Río Cuarto. Docentes Dr. Ing. Cristian de Ángelo, asistente: Mg. Lic. Laura Pérez. Duración 60 horas. Fecha: 11 de Septiembre de 2012 a 11 de Diciembre de 2012. Certificado: 27 de Mayo de 2013.}
\cvitem{2012}{TEORÍA DE LA POTENCIA INSTANTÁNEA Y APLICACIONES. Universidad Nacional de Río Cuarto. Segundo semestre 2012. Responsables: Dr. Ing. G. O. García y Dr. Ing. G.G. Oggier. Duración 60 horas. Aprobado. 03 de Noviembre de 2014.}
\cvitem{2013}{ELECTRÓNICA DE POTENCIA EN MICRORREDES CON ALTA PENETRACIÓN DE ENERGÍA RENOVABLE. (Ding-26) Universidad Nacional de Río Cuarto. 3 de Septiembre al 30 de Diciembre de 2013. Responsables: Dr. Ing. Denizar C. Martins y Dr. Ing. Guillermo. O. García. Duración 60 horas. Aprobado. Certificado: 29 de Diciembre de 2014. }
\cvitem{2015}{LABORATORIO DE IMPLEMENTACIÓN DE PROTOTIPOS EXPERIMENTALES. Universidad Nacional de Río Cuarto. Responsables: Dr. Ing. Germán G. Oggier y Dr. Ing. Guillermo. O. García. Duración 80 horas. Aprobado. 2015. }

\subsection{Seminarios de Posgrado}

\cvitem{2012}{SEMINARIO DE INVESTIGACIÓN EN CONTROL Y CONVERSIÓN DE LA ENERGÍA. Organizado por la Facultad de Ingeniería, Universidad Nacional de Río Cuarto. Duración 40 horas. Fecha: 25 de Abril de 2012 a10 de Diciembre de 2012. Certificado: 23 de Mayo de 2013. }
\cvitem{2011}{SEMINARIO DE INVESTIGACIÓN EN CIENCIAS DE LA INGENIERÍA. Organizado por la Facultad de Ingeniería, Universidad Nacional de Río Cuarto. Duración 40 horas. Fecha: 06 de Abril de 2011 a06 de Septiembre de 2011. Certificado: 20 de Noviembre de 2012. }
\cvitem{2011}{SEMINARIO DE INVESTIGACIÓN EN CONTROL Y CONVERSIÓN DE LA ENERGÍA. Organizado por la Facultad de Ingeniería, Universidad Nacional de Río Cuarto. Fecha23 de Marzo de 2011a 18 de Diciembre de 2011. Duración 40 horas. Certificado: 11 de Mayo de 2012. }


\subsection{Cursos de capacitación y perfeccionamiento}

\cvitem{2018}{Curso: ``Diseño Avanzado de sistema embebidos en lógica programable: Zynq APSoC, Vivado HLS y SDSoC''. Crédito Horario: 60hs.  LEIS. Electra Training. 19 al 23 de Marzo de 2018. Universidad Nacional de San Luis. San Luis, Argentina. }
\cvitem{2017}{Curso: ``Síntesis de Alto Nivel para FPGAs con Vivado-HLS''. LEIS. Electra Training. 11 y 12 de Abril de 2017. LEIS. Universidad Nacional de San Luis. San Luis, Argentina. }
\cvitem{2017}{Curso: Accelerating FPGA Desing (FPGA Competence). Bitbis AS (Norway).  8 y 9 de Mayo de 2017, Satellogic S.A. Buenos Aires, Argentina. }
\cvitem{2017}{Curso: Advanced VHDL Verification – Made Simple (FPGA Competence). Bitbis  AS (Norway). 9 al 11 de Mayo de 2017, Satellogic S.A. Buenos Aires, Argentina. }
\cvitem{2009}{Curso ``Advanced Training Course on FPGA Design and VHDL for Hardware Simulation and Synthesis'' en el Centro Internacional de Física Teórica (ICTP), en Trieste, Italia. Duración 120 hs reloj. Octubre, 2009}


\section{Antecedentes laborales en docencia}

\subsection{Categorización Docente Investigador}

\cvitem{}{Docente - Investigador ``Categoría V (cinco)''.}

\subsection{Docencia de Grado}

\cvitem{2017--presente}{Profesor Adjunto, Dedicación Exclusiva, Carácter Interino. Facultad de Ciencias Físico Matemáticas y Naturales. Universidad Nacional de San Luis. Fecha: 01 de Septiembre de 2017 al presente. Responsable de materias: Arquitectura de Computadoras (Ing. Electrónica con Orientación en Sistemas Digitales), Audio y Video (Tecnicatura en Telecomunicaciones) y Producción Multimedial (Tecnicatura en Telecomunicaciones).}
\cvitem{2016--2017}{Profesor Adjunto, Dedicación Exclusiva, Carácter Interino. Facultad de Ciencias Físico Matemáticas y Naturales. Universidad Nacional de San Luis. Fecha: 14 de Abril de 2016 al 30 de Agosto de 2017. Responsable de materias: Arquitectura de Computadoras (Ing. Electrónica con Orientación en Sistemas Digitales), Audio y Video (Tecnicatura en Telecomunicaciones) y Producción Multimedial (Tecnicatura en Telecomunicaciones).}
\cvitem{2013--2016}{Profesor Adjunto, Dedicación Simple, Carácter Temporario. Facultad de Ciencias Físico Matemáticas y Naturales. Universidad Nacional de San Luis. Fecha: 22 de Julio de 2013 al 13 de Abril de 2016. Responsable de materias: Arquitectura de Computadoras (Ing. Electrónica con Orientación en Sistemas Digitales), Audio y Video (Tecnicatura en Telecomunicaciones) y Producción Multimedial (Tecnicatura en Telecomunicaciones).}
\cvitem{2012--presente}{Ayudantía de Primera, Dedicación Simple, Carácter Efectivo. Facultad de Ciencias Físico Matemáticas y Naturales. Universidad Nacional de San Luis. Fecha: 01 de Noviembre de 2012 al presente. En licencia sin goce de haberes por incompatibilidad. Responsable de materias: Arquitectura de Computadoras (Ing. Electrónica con Orientación en Sistemas Digitales), Audio y Video (Tecnicatura en Telecomunicaciones) y Producción Multimedial (Tecnicatura en Telecomunicaciones).}
\cvitem{2011--2012}{Ayudantía de Primera, Dedicación Simple, Carácter Temporario. Facultad de Ciencias Físico Matemáticas y Naturales. Universidad Nacional de San Luis. Fecha: 1 de Abril de 2011 al 30 de Octubre de 2012. Responsable de Trabajos Prácticos: Arquitectura de Computadoras (Ing. Electrónica con Orientación en Sistemas Digitales) y Diseño de Sistemas Digitales (Ing. Electrónica con Orientación en Sistemas Digitales). }
\cvitem{2010--2011}{Ayudantía de Primera, Dedicación Exclusiva, Carácter Temporario. Facultad de Ciencias Físico Matemáticas y Naturales. Universidad Nacional de San Luis. Fecha: 12 de Marzo de 2010 al 31 de Marzo de 2011. Responsable de Trabajos Prácticos: Arquitectura de Computadoras (Ing. Electrónica con Orientación en Sistemas Digitales) y Diseño de Sistemas Digitales (Ing. Electrónica con Orientación en Sistemas Digitales).}
\cvitem{2009-2010}{Ayudantía de Segunda, Dedicación Simple, Carácter Temporario. Facultad de Ciencias Físico Matemáticas y Naturales. Universidad Nacional de San Luis. Fecha: 09 de Junio de 2009 al 11 de Marzo de 2010. Ayudante de Trabajos Prácticos: Arquitectura de Computadoras (Ing. Electrónica con Orientación en Sistemas Digitales) y Diseño de Sistemas Digitales (Ing. Electrónica con Orientación en Sistemas Digitales).}

\subsection{Docencia de Posgrado}

\cvitem{2013}{ELECTRÓNICA DE POTENCIA. Dictado por la Facultad de Ciencias Físico Matemáticas y Naturales, Universidad Nacional de San Luis. Docentes Dr. Ing. Germán Oggier. Auxiliar: Ing. Andrés Miguel Airabella. Responsable: Ing. Cristian Ariel Falco.  Duración 60 horas. Fecha: 01 de Noviembre de 2013 a 30 de Noviembre de 2013.}

\subsection{Evaluación de Trabajos Finales de Grado}

\cvitem{2016}{``Redes inalámbricas para el desarrollo de comunidades rurales digitales en el distrito de Yauya''. Autor: Ada Luz Caballero Sifuentes. Director: Ing. Alfredo Debattista. Carrera: Ingeniería Electrónica con Orientación en  Sistemas Digitales. UNSL. Diciembre 2016. }
\cvitem{2016}{``Sistema de medición con celda de carga, con excitación de Corriente Alterna, basado en amplificador tipo lock-in''. Autor: Laura Beatriz Adaro. Director: Carlos Federico Sosa Páez. Carrera: Ingeniería Electrónica con Orientación en  Sistemas Digitales. UNSL. Diciembre 2016. }
\cvitem{2011}{``Procesamiento de señales acústicas utilizando lógica programable''. Autor: Felix Leonardo Garro Martínez. Director: Diego Esteban Costa. Carrera: Ingeniería Electrónica con Orientación en  Sistemas Digitales. UNSL. 16 de Septiembre 2011.}
\cvitem{2011}{``Implementación de funciones básicas del amplificador Lock-in en FPGA''. Autor: Martínez Guevara, Layla María. Director: Ing. Peláez, Esteban Maximiliano. Carrera: Ingeniería Electrónica con Orientación en  Sistemas Digitales. UNSL. 19 de Diciembre de 2011.}

\subsection{Integrante de tribunales examinadores en concursos docentes}

\cvitem{2018}{Titular en Tribunal para concurso de cargo de AUXILIAR DE, PRIMERA CATEGORÍA, dedicación SIMPLE, carácter SUPLENTE con destino al ÁREA DE ELECTRÓNICA Y MICROPROCESADORES del Departamento de Física, Universidad Nacional de San Luis. 28 de marzo de 2018.}
\cvitem{2014}{Titular en Tribunal para concurso de cargo de AUXILIAR DE, PRIMERA CATEGORÍA, dedicación SIMPLE, carácter INTERINO con destino al ÁREA DE ELECTRÓNICA Y MICROPROCESADORES del Departamento de Física, Universidad Nacional de San Luis. 01 de septiembre de 2014.}

\subsection{Dirección y Co-Dirección de Becas y Trabajos Finales}

\cvitem{2016--2017}{Proyecto Final de Carrera: Alumno: Rodrigo Agustín Perna. Título del Plan de Trabajo: Sistema de supervisión y control en FPGA para un cargador solar de baterías. Año 2016-2017. En ejecución. Rol: Director. }
\cvitem{2016-2017}{Beca de Investigación del CIN (Consejo Interuniversitario Nacional): Alumno: Rodrigo Agustín Perna. Título del Plan de Trabajo: Sistema de supervisión y control en FPGA para un cargador solar de baterías. Año 2016-2017. En ejecución. Rol: Director. }
\cvitem{2016-2017}{Proyecto Final de Carrera: Alumno: Emiliano Álvarez. Título del Plan de Trabajo: Diseño de un Banco de Ensayos para Circuitos de Activación de Transistores de Potencia. Año 2016-2017. Rol: Director. }
\cvitem{2016--presente}{Proyecto Final de Carrera: Alumno: Rodrigo Sánchez. Título del Plan de Trabajo: Descripción en HDL de un detector de ángulo y secuencia positiva en sistemas eléctricos trifásicos. Año 2016-2017. En ejecución. Rol: Director.}
\cvitem{2012--2014}{Beca TIC 2012 y Proyecto Final de Carrera: Alumno: María Julia Xacur. Título del Plan de Trabajo: Desarrollo de instrumental específico para medir fuerza de ruptura de sólidos. Año 2012-2014. Finalizado. }
\cvitem{2012--2013}{Beca TIC 2012: Alumno: Gerardo Galo. Título del Plan de Trabajo: Diseño de una fuente para iluminación a LED robusta, de baja potencia. Año 2012-2013. Finalizado. }
\cvitem{2012--2013}{Co-Dirección de Trabajo Final de Carrera, Alumno: Enzo Belpoliti. Título del Plan de Trabajo: Implementación de la Técnica de Modulación SVPWM para Control de Inversores Trifásicos.  Año 2012-2013. Finalizado. }

\subsection{Dirección y Co-Dirección de Estancias de Investigación}
\cvitem{2012}{Dirección de Beca ``Verano de la Ciencia, Universidad Autónoma San Luis Potosí 2012''. Alumno: Oscar Zamarripa. Título del Trabajo: ``Operación de un Convertidor CC-CC con Puentes Duales Activos en modo Forward para el estudio de fallas en semiconductores''. Lugar: ``Grupo de Electrónica Aplicada – Facultad de Ingeniería– Universidad Nacional de Río Cuarto. Becario de Intercambio México-Argentina. Mayo a Junio de 2012.}
\cvitem{2013--2014}{Dirección de Beca de Intercambio ``UNSL-Hochschule Bonn Rhein Sieg''. Alumno: Sven Stockhausen. Título del Trabajo: ``Diseño de un instrumento para medir la fuerza de ruptura de sólidos''. Lugar: Laboratorio de Electrónica, Investigación y Servicios – Universidad Nacional de San Luis. Becario de Intercambio Alemania-Argentina. Octubre de 2013 a Febrero de 2014.}

\section{Antecedentes laborales en el sector privado}

\cvitem{2017--presente}{Mayo de 2017 al presente: Satellogic S.A. Ingeniero de Desarrollo en HDL (Hardware Description Language). }
\cvitem{2016--2017}{Agosto 2016 a Abril de 2017: Asesor del Gobierno de la Provincia de San Luis en Energías Renovables y Eficiencia Energética. Programa de Energías Renovables y Eficiencia Energética, Ministerio de Medio Ambiente, Campo y Producción. }
\cvitem{2009--2010}{Junio de 2009 a Abril de 2010: Becario del Grupo de Estudios Ambientales, para soporte en desarrollo de equipos electrónicos de medición de variables ambientales.}
\cvitem{2008--2009}{Octubre 2008 a Marzo de 2009: Diseño de módulos/macros digitales (IP Cores) optimizados para bajo consumo de potencia en lenguajes de descripción de hardware para la empresa MGB Design SRL, a su vez para Actel Corp (Ahora Microsemi).}


\section{Antecedentes en Investigación Científica y Desarrollo Tecnológico}

\subsection{Participación en Programas y Proyectos de Investigación y Desarrollo}
\cvitem{2016}{Proyecto de extensión universitaria "Fábricas Recuperadas por los Trabajadores". aceptado mediante la resolución CS 279-16 Universidad Nacional de San Luis. Función: Integrante.}
\cvitem{2015--2016}{``Equipo para medir dureza''. Proyecto del Programa Universidad, Diseño y Desarrollo Productivo 2015. Función: Director del proyecto. (Subsidio otorgador: \$25.000,00)}
\cvitem{2015--2016}{``Medidor Inteligente de Energía''. Proyecto del Programa Universidad, Diseño y Desarrollo Productivo 2015. Función: Director del proyecto. (Subsidio otorgador: \$25.000,00)}
\cvitem{2015--2017}{(Proyecto bianual) PDTS-CIN-CONICET PDTS209. ``Sistema modular de tracción para vehículos eléctricos''. Inv. Responsable: Cristian De Angelo,  Área: Ingeniería y Tecnología. Universidad Nacional de Río Cuarto. Subsidio otorgado \$199.800. El proyecto incluye una beca posdoctoral. CIN-CONICET, Res. CE Nº 1055-15.}
\cvitem{2015--2018}{(Proyecto trianual) PICT-2014-2760. ``Vehículos Urbanos de Tracción Eléctrica: Sistema de Propulsión y Gestión de Energía''. Inv. Responsable: Cristian De Angelo,  Área: Tecnología Energética Minera Mecánica y de Materiales. Universidad Nacional de Río Cuarto. Subsidio otorgado \$475.000. El proyecto incluye una beca de doctorado. ANPCyT, Res. Nº 270-15.}
\cvitem{2015--2018}{(Proyecto trianual). CONICET PIP 2014-2016 GI ``Vehículos Urbanos de Tracción Eléctrica: control, supervisión, gestión de energía e integración a la red eléctrica''. Director: García, Guillermo O. Co-director: De Angelo, Cristian H. Aprobado. Subsidio otorgado \$494.000. Res. 5013/14. }
\cvitem{2015--2018}{(Proyecto trianual) PICT-Start-Up 2014-3647 ``I+D de Tecnologías para UPS de Alta Potencia''. Director: Guillermo O. García. Proyecto conjunto entre CREXEL S.A. y el GEA-UNRC. Monto total otorgado \$600.000. }
\cvitem{2014--2015}{(Proyecto Bianual). PROIPRO 142514. ``Control de convertidores de potencia para sistemas de energías renovables''. Financiado por Universidad Nacional de San Luis. Línea de trabajo: ``Lógica programable aplicada a la electrónica de potencia''. Director del Proyecto: De Ángelo, Cristian. Director de la Línea: Falco, Cristian Ariel. }
\cvitem{2014--2017}{(Proyecto trianual) PICT-2013-1194 ``Paralelismo de inversores trifásicos para la integración de energías renovables en microrredes'' Temas Abiertos. García, Guillermo Oscar. Tecnología Energética, Minera, Mecánica y de Materiales, Universidad Nacional de Río Cuarto, Subsidio total + gastos de administración \$ 504.000.}
\cvitem{2014--2015}{``Banco de ensayo para drivers''. Proyecto del Programa Universidad, Diseño y Desarrollo Productivo 2014. Función: Director del proyecto. (Subsidio otorgador: \$24.987,00)}
\cvitem{2014--2015}{``PLD en electrónica de potencia'' Proyecto del Programa Universidad, Diseño y Desarrollo Productivo 2014. Función: Docente colaborador. (Subsidio otorgador: \$25.000,00)}
\cvitem{2012--2013}{``Energías alternativas para el suministro de energía eléctrica en la Región Centro Oeste''. Proyecto ANPCyT FONCyT CIN II PICTO-2010-0154 (subsidio otorgado \$200.000).}
\cvitem{2011--2013}{``Sistemas de Generación Electro-Eólicos - Escalas de 5 y 30 kW''. Proyecto conjunto entre el FONCyT, ANPCyT, PICT-2010, \#2744 (subsidio otorgado \$280.000). }
\cvitem{2011--2013}{``Detección y Diagnóstico de Fallas Incipientes en Accionamientos Eléctricos''. PIP CONICET 2010. Subsidio otorgado \$180.000. }
\cvitem{2013--2014}{``Control del flujo de energía eléctrica de cc tolerante a fallas''. Proyecto subsidiado por el Ministerio de Ciencia y Tecnología de la Provincia de Córdoba. Res. Nº 113/2011. Monto a otorgar \$40.000. Director: Dr. Ing. Germán Gustavo Oggier.}
\cvitem{2012--2015}{``Programa Control y Conversión de Energía''. Programa de Investigación subsidiado por la Secretaria de Ciencia y Técnica de la UNRC. Res. Rec. 328/12 (\$71.456). El Programa contiene a los siguientes Proyectos: Sistemas de Generación Electro-Eólicos, Vehículos Eléctricos e Híbridos, Diagnóstico de Fallas Incipientes en Accionamientos Eléctricos.}
\cvitem{2012}{Proyecto de Cooperación Internacional ``Topologías Electrónicas de Potencia para una Integración Eficiente de las Fuentes de Energías Renovables en Micro-Redes y Sistemas Eléctricos'', subsidiado por el Ministerio de Ciencia y Tecnología de la Provincia de Córdoba. Proyecto conjunto entre el GEA-UNRC, Argentina y INEP-UFSC, Brasil. Proyecto CCII -11347549-A-2011. Res. 174/11. }
\cvitem{2009--2011}{``Programa Control y Conversión de Energía''. Programa de Investigación subsidiado por la Secretaria de Ciencia y Técnica de la UNRC. Res. Rec. 442/09 (\$18.000). El Programa contiene a los siguientes Proyectos: Sistemas de Generación Electro-Eólicos, Vehículos Eléctricos e Híbridos, Diagnóstico de Fallas Incipientes en Accionamientos Eléctricos.}

\subsection{Becas Obtenidas}
\cvitem{2014--2015}{Beca Interna de Formación de Posgrado Tipo II. Proyecto: ``Convertidores CC-CC tolerantes a fallas para aplicaciones en sistemas eléctricos híbridos''. Otorgado por el Consejo Nacional de Investigaciones Científicas y Técnicas (CONICET). Fecha: Abril de 2014 a Marzo de 2016. }
\cvitem{2011--2013}{Beca Interna de Formación de Posgrado Tipo I. Proyecto: ``Convertidores CC-CC tolerantes a fallas para aplicaciones en sistemas eléctricos híbridos''. Otorgado por el Consejo Nacional de Investigaciones Científicas y Técnicas (CONICET). Fecha: Abril de 2011 a Marzo de 2014. }
\cvitem{2009}{International Center for Theoretical Physics (ICTP). Full Schoolarship para el curso ``Advanced Training Course on FPGA Design and VHDL for Hardware Simulation and Synthesis''. Trieste, Italy. October and November 2009. }
\cvitem{2007--2008}{Beca Estímulo de la Facultad de Ciencias Físico Matemáticas y Naturales. Trabajo de Investigación (Trabajo Final de Carrera): ``Procesamiento de Imágenes en FPGA''.}

\section{Producción en Investigación Científica - Publicaciones}


\subsection{Trabajos publicados en revistas internacionales indexadas}

\cvitem{2015}{A.M. Airabella, G.G. Oggier, L.E. Piris-Botalla, C. A. Falco, and G.O. García, ``Semiconductors Faults Analysis in Dual Active Bridge DC-DC Converter'' IET Power Electronics. Octubre-2015. DOI:  10.1049/iet-pel.2015.0299 , Online ISSN 1755-4543 }
\cvitem{2014}{L. E. Piris Botalla, G. G. Oggier, A. M. Airabella y G. O. García, ``Power Losses Evaluation of a Bidirectional Three-Port DC-DC Converter for Hybrid Electric System''. Elsevier International Journal of Electrical Power \& Energy Systems (ISSN: 0142-0615). Volume 58, June 2014, Pages 1-8}
\cvitem{2016}{L. E. Piris Botalla, G. G. Oggier, A. M. Airabella y G. O. García, ``Extensión del Rango de Operación con Conmutación Suave de un Convertidor CC-CC Bidireccional de Tres Puertos'' Revista Iberoamericana de Automática e Informática Industrial (RIAI) (ISSN: 1697-7912) - Aceptado Marzo de 2015.}

\subsection{Trabajos en prensa en revistas internacionales indexadas}

\subsection{Trabajos en evaluación en revistas internacionales indexadas}

\subsection{Trabajos publicados en revistas nacionales no indexadas}

\cvitem{2015}{A. M. Airabella, G. G. Oggier, L. E. Piris-Botalla, Cristian A. Falco y G. O. García, ``Estrategia de Detección de Fallas de Circuito Abierto en Semiconductores de Convertidores CC-CC Aislados'' Nota Técnica, Revista Ingeniería Eléctrica. Editores On-Line S.R.L. Marzo 2015. }

\subsection{Trabajos en evaluación en Congresos con referato nacional}

\subsection{Trabajos Completos en Congresos con referato internacional}

\cvitem{2018}{A. M. Airabella, C. Falco, G. Oggier y G. O. Garcia. ``Detección, Diagnóstico y Tolerancia a Fallas de Transistores en Convertidores CC-CC con Puentes Duales Activos en Modo Reductor''. IEEE ARGENCON 2018. 6, 7 y 8 de Junio de 2018. San Miguel de Tucumán}
\cvitem{2012}{A. M. Airabella, G. G. Oggier, L. E. Piris-Botalla, Cristian A. Falco y G. O. García, ``Open Transistors and Diodes Fault Diagnosis Strategy for Dual Active Bridge DC-DC Converter,'' 10th IEEE/IAS International Conference on Industry Applications (INDUSCON 2012). 5th to 7th of November of 2012 in Fortaleza.}
\cvitem{2012}{L.E. Piris-Botalla, G. G. Oggier, A. M. Airabella and G. O. García. ``Analysis and evaluation of power switch losses for three-port bidirectional DC-DC converter'' 2012 IEEE International Conference on Industrial Technology (ICIT),  Athens, Greece at 19-21 March 2012.}

\subsection{Trabajos Completos en Congresos con referato nacional}

\cvitem{2017}{M. E. Postemsky, S. F. Hernández Velázquez, R. M. Murdocca, A. M. Airabella. ``Medidor Inteligente de Energía''. VIII Congreso de Microelectrónica Aplicada. Córdoba 11 al 13 de octubre de 2017. }
\cvitem{2016}{R. A. Perna, O. E. Álvarez, J. R. Sánchez y A. M. Airabella. Lógica de Control de un Banco de Ensayos para Circuitos de Activación de Transistores de Potencia. VII Congreso de Microelectrónica Aplicada 2016. Universidad Nacional de San Luis, San Luis, Argentina. }
\cvitem{2016}{O. E. Alvarez, R. A. Perna, J. R. Sánchez y A. M. Airabella. Implementación de un banco de ensayos para circuitos de activación de transistores de potencia. VII Congreso de Microelectrónica Aplicada 2016. Universidad Nacional de San Luis, San Luis, Argentina. }
\cvitem{2016}{A. M. Airabella, G. G. Oggier, L. E. Piris-Botalla, C. A. Falco, G. O. García, ``Límites de Transferencia de Potencia de un Convertidor CC-CC con Puentes Duales Activos Tolerante a Fallas en Modo Reductor'', XVI Reunión de Trabajo en Procesamiento de la Información y Control (XVI RPIC) Ciudad de Córdoba, 6 al  8 de octubre de 2015. }
\cvitem{2015}{L. E. Piris-Botalla, G. G. Oggier, A. M. Airabella y G. O. García, ``Diseño de las Inductancias Auxiliares de un Convertidor CC-CC Bidireccional de Tres Puertos'' XVI Reunión de Trabajo en Procesamiento de la Información y Control (XVI RPIC) Ciudad de Córdoba, 6 al  8 de octubre de 2015. }
\cvitem{2015}{L. E. Piris-Botalla, G. G. Oggier, A. M. Airabella y G. O. García, ``Comparación de las Pérdidas de un Convertidor CC-CC Bidireccional de Tres Puertos para Diferentes Valores de Inductancias Auxiliares'' XVI Reunión de Trabajo en Procesamiento de la Información y Control (XVI RPIC) Ciudad de Córdoba, 6 al  8 de octubre de 2015.}
\cvitem{2015}{L. E. Piris-Botalla, G. G. Oggier, A. M. Airabella y G. O. García, ``Determinación y Extensión de los Límites de Operación de un Convertidor CC-CC Bidireccional de Tres Puertos'' XVI Reunión de Trabajo en Procesamiento de la Información y Control (XVI RPIC) Ciudad de Córdoba, 6 al  8 de octubre de 2015.}
\cvitem{2014}{E. Belpoliti, A. M. Airabella, Cristian Ariel Falco ``Técnica svpwm en fpga para el control de inversores trifásicos''. 24º Congreso Argentino de Control Automático. 27 al 29 de Octubre de 2014 – Buenos Aires, Argentina.}
\cvitem{2013}{A. M. Airabella, G. G. Oggier, L. E. Piris-Botalla, C. A. Falco, G. O. García, ``Esquemas Tolerantes a Fallas de Transistores de Potencia Para Convertidores con Puentes Duales Activos'', XV Reunión de Trabajo en Procesamiento de la Información y Control (XV RPIC) S.C. de Bariloche, 16 al  20 de setiembre de 2013.}
\cvitem{2011}{A. M. Airabella, G. G. Oggier, L. E. Piris-Botalla, Cristian A. Falco y G. O. García, ``Estrategia de Detección de Fallas de Circuito Abierto en Semiconductores de Convertidores CC-CC Aislados'' AADECA 2012, 3 al 5 de Octubre 2012, Buenos Aires, Argentina.}
\cvitem{2011}{A. I. Testa, M. R. Palavecino Nicotra, A. M. Airabella, F. Aguilera. ``Implementación de Sistema Embebido Sobre FPGA Basado en el Microprocesador miniMIPS''. III Congreso de Micro-Electrónica Aplicada 2012. Sección Estudiantil. 2012. ISBN 978-987-702-004-5.}
\cvitem{2011}{A. M. Airabella, G. G. Oggier, L. E. Piris-Botalla y G. O. García, ``Diagnóstico de fallas en semiconductores de potencia de Convertidores CC-CC con Puentes Duales Activos'' XIV Reunión en Procesamiento de la Información y Control (RPIC 2011), Universidad Nacional de Entre Ríos, Oro Verde, Entre Ríos, Argentina. Del 16 al 18 de noviembre de 2011.}
\cvitem{2011}{L. E. Piris-Botalla, G. G. Oggier, A. M. Airabella y G. O. García ``Convertidor CC-CC bidireccional de tres puertos: Evaluación de las pérdidas en las llaves de potencia'', XIV Reunión en Procesamiento de la Información y Control (RPIC 2011), Universidad Nacional de Entre Ríos, Oro Verde, Entre Ríos, Argentina. Del 16 al 18 de noviembre de 2011.}
\cvitem{2010}{A. M. Airabella, C. Sosa Páez, R. Petrino: ``Plataforma para Procesamiento de Imágenes en FPGA'', I Congreso de Microelectrónica Aplicada, UnLAM 2010. }


\section{Participación en Reuniones Científicas y Educativas}

\subsection{Como Autor de Trabajos}

\cvitem{2013}{XVI Reunión de Trabajo en Procesamiento de la Información y Control (XVI RPIC). Título del trabajo: ``Límites de Transferencia de Potencia de un Convertidor CC-CC con Puentes Duales Activos Tolerante a Fallas en Modo Reductor'', Ciudad de Córdoba, 6 al  8 de octubre de 2015. }
\cvitem{2012}{AADECA’12 – Semana del Control Automático. 23º Congreso Argentino de Control Automático. 3 al 5 de Octubre de 2012. Buenos Aires, Argentina. Título del Trabajo: ``Estrategia de detección de fallas de circuito abierto en semiconductores de convertidores CC-CC aislados''. }
\cvitem{2011}{XIV Reunión en Procesamiento de la Información y Control (RPIC 2011), Universidad Nacional de Entre Ríos, Oro Verde, Entre Ríos, Argentina. Del 16 al 18 de noviembre de 2011. Título del Trabajo: ``Diagnóstico de fallas en semiconductores de potencia de Convertidores CC-CC con Puentes Duales Activos''.}
\cvitem{2010}{I Congreso de Microelectrónica Aplicada. 2010. La Matanza, Buenos Aires, Argentina. Título del Trabajo: ``Plataforma para Procesamiento de Imágenes en FPGA''.}

\subsection{Como Conferencista, Docente u Organizador}

\cvitem{2018}{``Satellogic: Una charla sobre satélites''. 21 de Marzo de 2018, Universidad Nacional de San Luis. Organizada por Rama Estudiantil IEEE San Luis. }
\cvitem{2017}{SASE 2017 	- Disertante del Workshop: ``Punto de Partida para Diseños FPGA-VHDL'' en el Simposio Argentino de Sistemas Embebidos (SASE 2017), 9 al 11 de Agosto de 2017. }
\cvitem{2016}{uEA 2016 – VII Congreso de Microelectrónica Aplicada 2016. Universidad Nacional de San Luis. 26, 27 y 28 de Octubre de 2016, San Luis, Argentina. }
\cvitem{2015}{III JOREIC – 3eras Jornadas Regionales de Estudiantes de Ingeniería Civil 2015. ``¿Tenés dos minutos? Cómo transmitir ideas en tiempo límite''. Universidad Tecnológica Nacional. Facultad Regional San Rafael. 22 de Mayo de 2015.}
\cvitem{2014}{XX JOSEII. Título de la conferencia: ``¿Tenés dos minutos? Cómo transmitir ideas en tiempo límite''. Universidad Tecnológica Nacional. Facultad Regional San Rafael. 7 de Noviembre de 2014.}
\cvitem{2014}{Título de la conferencia: ``Prohibido Buscar Trabajo''. Universidad Nacional de Cuyo, Facultad de Ciencias Aplicadas a la Industria. Organizado por AECA. San Rafael. 4 de Noviembre de 2014.}
\cvitem{2014}{II Semana Nacional del Emprendedor Tecnológico. Título de la conferencia: ``Prohibido Buscar Trabajo''. Universidad Tecnológica Nacional. Facultad Regional San Rafael. 16 de Septiembre de 2014. }
\cvitem{2012}{Séptimo Encuentro Nacional de Estudiantes de Ingeniería (ENEI 2012). Título de la conferencia: ``¿Ingeniero? Prohibido Buscar Trabajo''. 18 y 19 de Agosto de 2012. San Luis, Argentina. }
\cvitem{2011}{Simposio Argentino de Sistemas Embebidos 2011. Buenos Aires, Argentina. Título de la conferencia: ``Simulación Avanzada con TestBench en HDL''.}
\cvitem{2010}{Quinto Encuentro Nacional de Estudiantes de Ingeniería (ENEI 2010). Título de la conferencia: ``¿Ingeniero? Prohibido Buscar Trabajo''.25, 26  y 27 de Noviembre de 2010, Río Cuarto, Córdoba, Argentina. }
\cvitem{2010}{I Congreso de Microelectrónica Aplicada. 2010. La Matanza, Buenos Aires, Argentina. Título de la conferencia: ``Simulación Avanzada con TestBench en HDL''.}
\cvitem{2010}{Seminario ``Introducción a las Tecnologías de Lógica Programable'', realizado en la Universidad Nacional del Comahue, Octubre 2010. Contenidos: Diseño digital usando VHDL para síntesis y simulación.}
\cvitem{2009}{Seminario ``Introducción a las Tecnologías de Lógica Programable'', realizado en la Universidad Nacional de Catamarca, los días 12, 13 y 14 de agosto de 2009, organizado por la rama estudiantil del IEEE de dicha universidad. Contenidos: Diseño digital usando VHDL para síntesis y simulación.}
\cvitem{2011}{Docente-Tutor de Laboratorio en el curso ``ICTP Latin-American Basic Course on FPGA Design for Scientific Instrumentation'', organizado por el Centro Internacional de Física Teórica (ICTP), en Mar del Plata, Buenos Aires, Argentina. Febrero 2011. }

\subsection{Como Asistente}

\cvitem{2011}{Ciclo de Conferencia sobre Energía Eólica 2011. Organizado por: Grupo de Electrónica Aplicada, Grupo de Análisis de Sistemas Eléctricos de Potencia, Capítulo Conjunto Argentino de IEEE y Capítulo Argentino IEEE PES. 16 de Julio de 2011. Río Cuarto, Córdoba, Argentina. }


\section{Premios y distinciones}

%\cvitem{2008}<Abanderado año 2008, portador de bandera provincial. 
%\cvitem{2011}{School of Business Postgraduate Scholarship}
%\cvitem{2010}{Top Achiever Award -- Commerce}

%----------------------------------------------------------------------------------------
%	COMPUTER SKILLS SECTION
%----------------------------------------------------------------------------------------

\section{Herramias informáticas y de programación}



\section{Experiencia en Liderazgo}

\subsection{IEEE Young Professionals Programm}

\cvitem{2016}{Presidente de IEEE Young Professionals Program de la Sección Argentina.}
\cvitem{2015}{Vicepresidente de la Sección Argentina de IEEE Young Professionals Program.}
\cvitem{2014}{Secretario de la Sección Argentina de IEEE Young Professionals Program.}

\subsection{Rama Estudiantil del IEEE}

\cvitem{2011-2012}{Miembro de la Rama Estudiantil del IEEE de la Universidad Nacional de Río Cuarto.}
\cvitem{2007}{Former President 2007. Rama Estudiantil IEEE UNSL}
\cvitem{2006}{Presidente 2006. Rama Estudiantil IEEE UNSL. Organizador ``Reunión Nacional de Ramas Estudiantiles del IEEE 2006''. Ciudad de San Luis, del 16 al 18 de Noviembre de 2006.}
\cvitem{2005}{Coordinador de la Comisión Sitio Web de la Rama Estudiantil IEEE San Luis.}

%\cvitemwithcomment{English}{Mothertongue}{}
%\cvitemwithcomment{Spanish}{Intermediate}{Conversationally fluent}
%\cvitemwithcomment{Dutch}{Basic}{Basic words and phrases only}

\section{Intereses}

\renewcommand{\listitemsymbol}{-~} % Changes the symbol used for lists

%\cvlistdoubleitem{Piano}{Chess}
%\cvlistdoubleitem{Cooking}{Dancing}
%\cvlistitem{Running}

%----------------------------------------------------------------------------------------

\end{document}
